\chapter{Conclusion}
\label{chap:conclusion}

In this chapter, I discuss potential improvements to the two-sided matching
model of FDI. In addition, I explore other areas of Political Science that the
two-sided matching model may be applicable. Finally, I conclude with final
thoughts about how the FDI literature can better contribute to the pressing
issue of globalization.

\section{Potential improvements}

In Chapter~\ref{chap:FDI}, the analysis uses FDI location data from only one
year. Instead, we can use the entire panel of data to estimate countries'
preference. To do so, we add one more level to our hierarchical model. In this
new setup, the lowest level is a country-year, whose preference parameters are
drawn from a normal distribution centered on the preference matters of the
corresponding country. With this model, we can study how countries' preference
evolve over time, a phenomenon that qualitative case studies in
Chapter~\ref{chap:literature_issues} and Chapter~\ref{chap:FDI} have
demonstrated.

In addition, we can ``explain'' countries' preference by building a regression
model after we have estimated their preference. In this model, the dependent
variable is the estimate for countries' preference and the independent variables
are factors such as regime type or the government's time horizon. For example,
we may hypothesize that governments with a longer time horizon attract more R\&D
heavy FDI because they are likely to be in power long enough to reap the rewards
of R\&D. Currently, such a regression model does not have much power due to the
small sample size of countries. If we could expand the model to analyze panel
data, in which each country-year is one observation, we would have more
statistical power to study what factors shape countries' preference for FDI.

If we had data on the specific offers that MNCs receive from countries, we can
also produce more precise estimate of MNCs' preference. As the case studies of
Korea and Taiwan in Chapter~\ref{chap:literature_issues} demonstrate, countries
and MNCs engaged in intense negotiation, making tailored offers and specific
requirements. While it is unlikely to systematically collect such a detailed
dataset of individual deals, data on fiscal incentives, especially tax holiday,
are sometimes available in business surveys.\footnote{For example, Vietnam's
  Provincial Competitiveness Index (PCI) routinely asks foreign firms about the
  incentives they receive from Vietnam's provincial governments.}

\section{Other applications for two-sided matching model}

\subsection{US federal clerkship market}

In the US, graduates at top law schools vie for the best federal clerkship every
year. These temporary, one-to-two-year positions are the launching pad for
Supreme Court clerkship, prestigious teaching jobs, or employment at top law
firms. On the other side, federal judges also compete for the best law
graduates, who help reduce the judges' workload, ranging from copy-editing to
drafting opinions \citep{Gulati2016, Posner2001}. Because the first clerkship
tends to have an outsized ideological influence on law graduates, this matching
market has important implications for the polarization of the judicial branch
\citep{Ditslear2001, Liptak2007}.

The market for US federal clerkship has been noted as a classic case of a
two-sided market. Clerks look for positions that provide not only prestige and
connection but also comfortable quality of life \citep{Posner2001}. Judges
select law graduates based on not only academic credentials but also ideology,
gender, and race \citep{Slotnick1984}. This market also suffers from strategic
behavior emblematic of a matching market, such as offers being made aggressively
early and with a short time to accept \citep{Posner2001, Posner2007}.

One approach to estimating the preference of agents in this market is to survey
clerks and judges directly \citep{Peppers2008}. However, this approach only
measures stated preference, which is likely to suffer from social desirability
bias when it comes to dimensions that we care about most, e.g. matching based
on ideology, gender, or race.

Other approaches estimate revealed preference by using observed hiring outcome.
However, no existing study has properly taken into account the two-sided nature
of the market, thus confusing the effects of preference and opportunity. For
example, \citet{Bonica2017} use political contribution data (DIME dataset) to
measure political ideology, then correlate the ideology of the hiring judge and
the ideology of his clerks. This approach does not take into account the pool of
applicants, leading to conclusions such as conservative judges hire more liberal
clerks than conservative clerks \citep[31]{Bonica2017}. This curious finding has
a potentially simple explanation: the pool of top law graduates tend to be
overwhelmingly liberal, leaving conservative judges with no choice. Despite
this issue, the authors proceed to measure judges' ideology by taking the
average of their clerks' ideology. Without taking the pool of applicants into
account, they may wrongly conclude that conservative judges are more liberal
than they actually are.

In another approach, \citet{Rozema2016} model the process as a discrete choice
problem, in which clerks are differentiated products that Supreme Court justices
select to maximize their utilities. Their model does not need to consider what
clerks think about the offer because they focus on Supreme Court clerkship,
whose unparalleled prestige ensures that any offer made will be accepted.
However, if we want to extend the model to the broader market of federal
clerkship, such assumption is untenable.

\subsection{The market for forming a coalition government}

Besides election, government formation is the most consequential political
process in determining which government citizens are subject to. Most extant
studies of government formation are either game theoretic models or thick,
``inside-the-Beltway'' narratives. We can potentially advance the literature by
considering government formation as a many-to-one matching market, with the
\textit{formateur} party on one side and other minority parties on the
other.\footnote{The \textit{formateur} party could be the one with the
  procedural power to set up the coalition, e.g. the incumbent party, or the
  largest party in established coalitions.}

A two-sided matching model of government formation would complement the game
theory literature that models politicians as policy-seeking (as opposed to
office-seeking) \citep{Laver1998}. When politicians are policy-seeking, parties
have policy positions that can be modeled as their characteristics. Then,
parties choose one another to form a coalition based on their policy positions,
akin to men and women choosing one another to form a marriage based on their
height or income.\footnote{In contrast, when politicians are office-seeking, the
  only coin of the realm is the number of legislative seats that a party
  controls. It determines both the inclusion of the party in the government and
  its portfolio allocation. In this framework, concepts like power indices and
  dominant parties are all about how parties can turn a coalition into a winning
  one by using its controlled seats. The two-sided matching model is not
  suitable for this case because parties are not looking for their policy
  match.} As the game theory literature suggests, ideologically compact
coalitions are more valuable because they entail a smaller cost in terms of
policy compromises \citep{DeSwaan1973}. With the empirical matching model, we
can test if parties do indeed prefer others that are ideologically close to
themselves.

In addition, an advantage of the two-sided matching approach is its ability to
consider multidimensional policy spaces. By considering a party's positions on
various policies as their covariates, we would be able to estimate parties'
relative preference for ideological proximity across policy dimensions.

\subsection{Final thoughts}

Once a match is formed in a two-sided matching market, the two involved parties
are committed and no longer available to others on the market. Therefore,
matching markets tend to involve weighty decisions: marriage, job, organ
donation, or government formation. To study matching markets is to examine some
of the most consequential social processes.

With FDI inflow making up 9.4\% of the global fixed capital formation in 2016,
the FDI market is one such consequential process \citep{UNCTAD2017}. The
importance of FDI attracts substantial attention from IPE scholars, yet none has
paid attention to its two-sided nature and to the preference of countries in
this market. Such neglect is surprising given that political scientists are
first and foremost interested in politics. The formation of countries'
preference is inherently political, and should be of interest to our field.
Perhaps the inattention to the preference of countries in the FDI market stems
mainly from the methodological challenge of analyzing a two-sided market. If
that is the case, I hope that my research has made it less of a barrier.

If future FDI research pays more attention to countries' preference, the benefit
extends far beyond the ability to come up with better estimates. More
importantly, the FDI literature will be able to speak to the broad and important
issue of government policies in the era of globalization. What policies will
countries adopt in reaction to global capital? And which domestic
constituencies will shape their policies?

These questions are not new. The 1999 Seattle Protests surrounding the WTO
Ministerial Conference were a physical and violent embodiment of these concerns.
Labor unions protested the outsourcing of jobs, environmentalists fought MNCs'
pollution, labor rights activists confronted working conditions in third world
factories---different groups had different enemies, but everyone was connected
in the collective cause of protecting local politics from global interest and
people from corporations.

These questions are not pass\'e either. In 2007, political scientist Kenneth
Scheve and economist Matthew Slaughter warned that less-skilled US workers were
increasingly anxious about being the losers of globalization, causing
protectionism to be on the rise. In response, they called for ``A New Deal for
globalization,'' which would rebuild the support for globalization by
compensating those hurt in the process \citep{Scheve2007}. Their call went
unheeded, and their warning came true. The US 2016 election was in large part a
referendum on globalization, and large swaths of American voters said, ``Not for
me.''

In a role reversal, it is developing countries that are now enthusiastic
believers, providing ``the strongest support across the board for foreign
investment, trade and the benefits to be derived from
globalization.''\footnote{http://www.pewglobal.org/2014/09/16/faith-and-skepticism-about-trade-foreign-investment/}
As beneficiaries, developing countries will likely maintain their strong support
for FDI in particular and globalization in general. At the same time, their
challenge is to upgrade the quality of their FDI and build up their domestic
business, allowing them to participate in a higher value added step in the
global value chain. It remains unclear how to do so without a market and cash reserve
the size of China's.

In conclusion, I hope that my two-sided approach to the FDI market has not only
sparked scholars' interest in countries' FDI policy but also provided the tool
to study it. I have taken the first step in this research agenda by estimating
countries' preference---the next challenge is to study their trends and examine
their determinants. I am excited to see future developments in this field.

%%% Local Variables:
%%% mode: latex
%%% TeX-master: "AnhLe_dissertation.tex"
%%% End: