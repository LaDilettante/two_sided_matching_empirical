\chapter{Conclusion}
\label{chap:conclusion}

\section{Potential improvements}

On the estimation of countries' preference, instead of using cross-sectional
data of FDI location in only one year, we can leverage the entire panel of data.
To do so, we may add one more level to our hierarchical model. In this new
setup, the lowest level is the preference of a country-year, which is drawn from
a Normal distribution whose mean is the preference of a country. By doing so, we
can study how countries' preference evolve over time, which qualitative case
studies in Chapter~\ref{chap:literature_issues} and Chapter~\ref{chap:FDI} have
demonstrated.

In addition, we can ``explain'' countries' preference by building a regression
model in which the dependent variable is the estimates for countries' preference
and the independent variables are factors such as regime type or the
government's time horizon. We may hypothesize that governments with a longer
time horizon attract more R\&D heavy FDI because they are more likely to be in
power to reap the reward of the productivity increase from R\&D. Currently, such
a regression model does not have much power due to the small sample size of
countries. If we could expand the model to analyze panel data, in which each
country-year is one observation, we would be more able to detect how political
factors shape countries' preference for FDI.

On the estimation of MNCs' preference, we can produce more precise estimates if
we had data on the specific offers than MNCs receive from countries. As the case
studies of Korea and Taiwan in Chapter~\ref{chap:literature_issues} demonstrate,
countries and MNCs engaged in intense negotiation with tailored offers and
specific requirements. While it is unlikely to systematically collect and
analyze all the details, data on fiscal incentives, especially tax holiday, are
sometimes available in business surveys.\footnote{For example, Vietnam's
  Provincial Competitiveness Index (PCI) routinely aske foreign firms about the
  incentives they receive from Vietnam's provincial governments.}

\section{Other applications for two-sided matching model}

\subsection{US federal clerkship market}

In the US, graduates at top law schools vie for the best federal clerkships
every year. These temporary, one-to-two-year positions are the launching pad for
Supreme Court clerkships, prestigious teaching jobs, or employment at top law
firms. On the other side, federal judges also compete for the best law
graduates, who help reduce the judges' workload from copy-editing to drafting
opinions \citep{Gulati2016, Posner2001}. Because the first clerkship tends to
have an outsized ideological influence on law graduates, this matching market
has important implications for the polarization of the judicial branch
\citep{Ditslear2001, Liptak2007}.

The market for US federal clerkship has been noted as a classic case of a
two-sided market. Clerks look for positions that provide not only prestige and
connection but also comfortable quality of life \citep{Posner2001}. Judges
select law graduates based on not only academic credentials but also, some
argue, ideology, gender, and race \citep{Slotnick1984}. This market also suffers
from strategic behavior emblematic of a matching market, such as offers being
made aggressively early and with a short time to accept \citep{Posner2001,
  Posner2007}.

One approach to estimating the preference of agents in this market is to survey
clerks and judges directly \citep{Peppers2008}. However, as discussed, this
approach only measures stated preference, which is likely to suffer from social
desirability bias when it comes to dimensions that we care about most such as
matching based on ideology, gender, or race.

Other approaches estimate revealed preference by using observed hiring outcome.
However, no existing study has properly taken into account the two-sided nature
of the market, thus confusing the effects of preference and opportunity. For
example, \citet{Bonica2017} use political contribution data (DIME dataset) to
measure political ideology, then correlate the ideology of the hiring judge and
the ideology of his clerks. This approach does not take into account the pool of
applicants, leading to conclusions such as conservative judges hire more liberal
clerks than conservative clerks \citep[31]{Bonica2017}. This curious finding has
a potentially simple explanation: the pool of top law graduates tend to be
overwhelmingly liberal, leaving conservative judges with no choice. Despite
this issue, the authors proceed to measure judges' ideology by taking the
average of their clerks' ideology. Without taking the pool of applicants into
account, they may wrongly conclude that conservative judges are more liberal
than they actually are.

In another approach, \citet{Rozema2016} model the process as a discrete choice
problem, in which clerks are differentiated products that Supreme Court justices
select to maximize their utilities. Their model does not consider what clerks
think about the offer because of their focus on Supreme Court clerkships, whose
unparalleled prestige ensures that any offer made will be accepted. However, if
we want to extend the model to the broader market of federal clerkship, such
assumption is untenable.

\subsection{The market for forming a coalition government}

Besides election, government formation is the most consequential political
process in determining which government people are subject to. Most extant
studies of government formation are either game theoretic models or thick,
``inside-the-Beltway'' narratives. Potential advances can be made if we consider
government formation as a many-to-one matching market, with the
\textit{formateur} party on one side and other minority parties on the
other.\footnote{The \textit{formateur} party could be the one with the
  procedural power to set up the coalition, e.g. the incumbent party, or the
  largest party in established coalitions.}

A two-sided matching model of government formation would complement the game
theory literature that models politicians as policy-seeking (as opposed to
office-seeking) \citep{Laver1998}. When politicians are policy-seeking, parties
have policy positions that can be modeled as their characteristics. Then,
parties choose one another to form a coalition based on their policy positions,
akin to men and women choosing one another to form a marriage based on their
height or income.\footnote{In contrast, when politicians are office-seeking, the
  only coin of the realm is the number of legislative seats that a party
  controls. It determines both the inclusion of the party in the government and
  its portfolio allocation. In this framework, concepts like power indices and
  dominant parties are all about how parties can bring its controlled seats to a
  coalition to turn it into a winning coalition.} As the game theory literature
suggests, ideologically compact coalitions are more valuable because they entail
a smaller cost in terms of policy compromises \citep{DeSwaan1973}. With the empirical
matching model, we can test if parties do indeed prefer others that are
ideologically close to themselves.

In addition, an advantage of the two-sided matching approach is its ability to
consider multidimensional policy spaces. By considering a party's positions on
various policies as their covariates, we would be able to estimate parties'
relative preference for ideological proximity across policy dimensions.
%%% Local Variables:
%%% mode: latex
%%% TeX-master: "AnhLe_dissertation.tex"
%%% End: