
Relative exchange rate (Takagi2011). Assume that the capital market is
imperfect, that external borrowers face a premium. So if a host country currency
depreciates, inflow FDI will increase because the same amount of source currency
can buy more input in the host country. The relationship is especially strong
between exchange rate and inflow of FDI into industries with a lot of
firm-specific assets. (e.g. Japanese firms buy a US firm with innovation for
cheap, then use that innovation to improve its production back home in yen)

(Originally people don't think exchange rate matter because if you can buy an
asset for cheap in the host country, when you repatriate the profit back to the
home country it's a wash)

FDI is all about the relative factors (because a firm thinks about a country in
terms of how that country is relative to its host). So the two sided matching
framework makes sense

General FDI: Why don't firms export or license, but open their own plant
overseas? Argument is that they have firm-specific asset that cannot be fully
exploited otherwise. A licensee won't be able to exploit the entire asset, both
sides can't agree on a price beforehand (This is the OLI framework,
ownership-location-internalization, also the internationalization hypothesis).
Empirically, firm specific asset is unobservable, so people use R\&D intensity
and advertising intensity instead. R\&D does correlate strongly with
multinationality.
  
\item Democracy: Democracy has been a mainstay in the political science
  literature on FDI. Scholars have argued that MNCs want to invest in democratic
  regimes for various reasons, including stable policy, credible commitment, and
  strong property rights \citep{Ahlquist2006, Li2003, Jensen2003}. On the other
  hand, recent works have also argued that democratic regimes want FDI more than
  autocratic regimes \citep{Pandya2016}. Thus, it is unclear whether the
  observed high level of FDI in democracies is due to the push or the pull
  factors. By controlling for countries' preference in the two-sided matching
  model, I can better estimate the effect of democracies on firms' utility. I
  measure democracy using the binary Demoracy \& Dictatorship, developed by
  \citet{Cheibub2009b}.