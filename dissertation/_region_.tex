\message{ !name(AnhLe_dissertation.tex)}\documentclass[]{dukedissertation}
%\documentclass[economy,twoside,bind]{dukedissertation}
% Use the second for a single-spaced copy suitable for duplex printing
% and binding.

% Other useful options (there are more options documented in Chapter 2):
%  * draft -- don't actually include images, print a black bar on overful
%             hboxes
%  * MS    -- Format for a Master's Thesis.  No UMI abstract page, some 
%             textual changes to title page.  


% Useful packages for dissertation writing:
\usepackage{amsmath, amssymb, amsfonts, amsthm}
\usepackage{graphicx}
\usepackage{natbib}
\usepackage{color}
\usepackage{bm}
\usepackage{subfigure}
\usepackage{graphicx}
\usepackage{mathabx}
\usepackage{multirow}
\usepackage{setspace}
% \usepackage{cite}  % If you include this, hyperlink cites will
                     % break.  It's nice to use this package if your bibstyle
							% sorts entries by order-of-use, rather than
							% alphabetically (as plain does).
							
%Theorem, Lemma, etc. environments
\newtheorem{theorem}{Theorem}%[section]
\newtheorem{lemma}[theorem]{Lemma}
\newtheorem{proposition}[theorem]{Proposition}
\newtheorem{corollary}[theorem]{Corollary}
\newtheorem{result}[theorem]{Result}

% Personal commands and abbreviations.
%Define and personal commands here

%Graphics Path to find your pictures
\graphicspath{{./Pictures/}{../figure/}}


%-----------------------------------------------------------------------------%
% PREAMBLE 
%-----------------------------------------------------------------------------%
\author{Anh Le}
\title{Two-Sided Matching Model}
\supervisora{Michael Ward}
\supervisorb{Eddy Malesky}
\department{Department of Political Science} % Appears as Department of \department
% Declare dissertation subject used on UMI abstract page.  List of
% categories: http://dissertations.umi.com/duke/subject_categories.html
%\subject{[Your Subject Here]}

\date{2018} % Anything but the year is ignored.

% Copyright text.  If undefined, default is 'All rights reserved'
% (Example sets the text to a hyperlinked Creative Commons Licence)
\copyrighttext{ All rights reserved except the rights granted by the\\
   \href{http://creativecommons.org/licenses/by-nc/3.0/us/}
        {Creative Commons Attribution-Noncommercial Licence}
}

% Committee Members other than supervisor.  No more than five beyond the
% supervisor allowed.
\member{Daniel Stegmueller}
\member{Peter Hoff}
%-----------------------------------------------------------------------------%


%-----------------------------------------------------------------------------%
% HYPERREF: plain black hypertext references for ref's and cite's.
%-----------------------------------------------------------------------------%
\usepackage[pdftex, pdfusetitle, plainpages=false, 
				letterpaper, bookmarks, bookmarksnumbered,
				colorlinks, linkcolor=black, citecolor=black,
	         filecolor=black, urlcolor=black]{hyperref}

\begin{document}

\message{ !name(fdi.tex) !offset(-57) }
\section{Data}

The dataset was compiled by Andrew Delios from the \textit{Kaigai Shinshutsu
  Kigyou Souran} (Japanese Overseas Investments-by Country), 1986-1999 editions,
a biennial publication that contains operational data on all foreign affiliates
of Japanese firms.\footnote{I thank Professor Andrew Delios for generously
  sharing the data.} Tokyo Keizai, Inc. collects this data via a survey of these
affiliates, which is reputed to include all Japanese firms overseas
\citep{Yamawaki1991}. Comparing the Japanese Overseas Investment data with other
data sources on publicly listed firms, \citep{Delios2001} finds that the dataset
covers 98.5\% of public firms, which in turn control 99.5\% of the foreign
subsidiaries. This high level of coverage ensures that our data captures the
entire set of options available to countries and firms, obviating any worry
about whether the choice set in the data represents the choice set in
reality.\footnote{The mismatch between the choice set in the sample and in the
  population is an unexplored theoretical aspect of two-sided matching models.
  Consider an example where we analyze a sample of 1000 men and women in the US
  to estimate their mate preferences. How are our estimates affected by our
  assumption that the potential choice set of each man includes all the women
  (and vice versa)? Not only does an individual not have that many
  acquaintances, his social circle is also not a representative sample of the
  entire dataset \citep[568]{Logan2008}. Fortunately, this is not a problem for
  our application. Given that there are only 9 Asian investment locations and
  approximately 200 Japanese subsidiaries being built each year, we can
  reasonably assume that they are all available to one another as potential
  options.}

From this dataset, I make several choices restricting the sample to better fit
with the assumption of the two-sided matching model.

First, I limit the sample to subsidiaries that are founded in the year 1996. The
reason to limit the sample to subsidiaries founded in a particular year instead
of including subsidiaries that have already invested is because the MNCs'
utility function in our model does not capture the fixed cost of relocating.
Indeed, as a linear combination of only the country covariates, the utility
function does not take into account the fact that a subsidiary may not relocate
to a new country even if country B is a better option. Therefore, the utility
function in our model is not appropriate for subsidiaries that have already
invested, and should be best used to model only subsidiaries that are While past
applications of the two-sided matching model do use a cross-sectional This is
important for FDI because, unlike equity investor, FDI are less foot-loose.
Indeed, the fact that it is not footloose is an important quality of FDI that's
appealing to countries (cite). The political economy literature on FDI has also
derived its insights largely from this ``obsolescing bargain'' problem, so it's
important that our model takes this into account. In past applications of the
two sided matching approach, researchers use a random sample in time (i.e. a
sample of all couples who's married in a certain year), which is fine, because
the cost of leaving a job or leaving a partner may not be too onerous. However,
for FDI, this fixed cost is more central. By limiting the sample to the firms
who are founded in 1996, we examine their decisions as they are all looking for
potential locations.

+ the data is largest for this year. (some summary statistics here). There may
be some concerns about this year being a special year, in the year leading up to
the 1997 Asian Financial Crisis. But 1) we're only looking at manufacturing
firms, not equity investors or land developers, 2) FDI during the crisis is
largely the same as before the crisis \citep{UNCTAD1998}. Indeed, this is
because FDI firms are largely looking at countries' fundamentals, such as labor
cost, market potential, and thus not affected by the fluctuations in the
financial markets. Essentially, the types of firms that invest before, during,
and after crisis are still the same types of firms.\footnote{One potential
  concern is that our data does not capture the firms who thought about making
  an investment but decided not to. This could be a problem if the MNCs in our
  dataset is the most risk-seeking firms, then essentially we've only estimated
  the preference of the very risk-seeking or the very risk-averse firms. AFC too
  high short term interest, too high local currency that is fixed. However, the
  exit rate of Japanese firms in Thailand, the epicenter of the financial
  crisis, is the same, indicating that the types of firms who invest are not so
  affected by the financial crisis. Looking at the exit is the reverse way of
  looking at who would have invested but didn't. \citep{Delios2001}}

(I use manufacturing FDI, so most capital in manufacturing is fixed) (because of
the the ``obsolescing bargaining'' argument. )

+ I only consider Japanese FDI into East Asian and Southeast Asian economies to
make sure that it's realistic to say that all these companies have the same
preference parameters. \citep{Pak2005} finds that Japanese FDI in the West seeks
to augment their global competitiveness, while Japanese FDI in the East focuses
on exploiting their core competencies. Japanese FDI in the West are ones with
oligopolistic power in their domestic market (so they are in a strong position
to compete) and require R\&D and marketing capabilities. They have different
level of equities. This suggests that the two types of firms are fundamentally
different.\footnote{Even though the difference in theory could be due to the
  preference of the countries. However, it doesn't make sense why the level of
  Japanese ownership is also different (because countries would not care about
  this).}

The final sample includes 6474 Japanese foreign affiliates in 2003, spreading
across 37 countries, with China and the US leading as the two top destinations
for Japanese MNCs (Table \ref{tab:list_of_countries}).

For firms' characteristics that countries consider, I include:

\begin{itemize}
\item Capital size (in US\$): A main argument for the benefit of FDI is that it
  brings capital to the country, improving labor productivity. MNCs' capital is
  especially important for developing countries, which cannot muster much
  domestic capital from their poor population. The capital size of a firm is
  included in the Japanese Overseas Business dataset.

\item Labor size: Similarly, a reputed benefit of FDI is that it creates jobs,
  generating not just economic growth but also increasing the government's
  popularity among the populace. The total number of employees of a firm is
  included in the Japanese Overseas Business dataset.

\item Technology intensity: I proxy for a firm's technology intensity by the
  industry to which it belongs. \citet{OECD2009} categorizes ISIC industries
  into four levels of technology intensity---low, medium low, medium high, and
  high---according to the level of R\&D expenditure divided by sales. I convert
  the industry classification of firms in my data from SIC 3 to ISIC and
  categorize their technology intensity from 1 to 4, with 1 being low and 4
  being high. On several occasions, one industry in SIC 3 matches to multiple
  ISIC (rev 3) industries or none at all. In the former case, I take the average
  across matched ISIC industries. In the latter case, the data is missing and
  later removed from the analysis.\footnote{\cite{Bergstrand2007} discusses the
    difference between R\&D intensity and advertising intensity, and find that
    R\&D intensity is higher for manufacturing firms compared with consumer
    product firms. Plus R\&D intensity is much more important for firms'
    performance than advertising intensity.}\footnote{Definition of R\&D
    intensity: the amount spent on R\&D as a percentage of sale}

\item Export intensity (ratio of export to sale):
\end{itemize}

Summary statistics table for these covariates.


For countries' characteristics that firms consider, I include:

\begin{itemize}
\item Market size: MNCs are expected to prefer countries with a large market
  size, which present MNCs with many potential customers. Indeed, this has been
  often cited as the allure of China to MNCs \citep{Luo2010}, as well as
  confirmed in larger studies. This is also a major variable in the gravity
  model, which has become a standard model for analyzing FDI flows.
  \citet{Bergstrand2007} provides the theoretical framework for the use of
  gravity model. I follow the standards in the literature and include log GDP
  (constant 2005 US\$), taken from the Penn World Table.\footnote{An advantage
    of the Penn World Table is that it compiles data for Taiwan, an important
    destination that the World Bank Development Indicators does not include.}

\item Level of development: MNCs are expected to prefer countries with a high
  level of development. A developed economy has consumers with high purchasing
  power and better infrastructure. It can also measure capital abundance, in
  which case a higher GDP per capita imply less flow because the simple model of
  FDI frames FDI as the movement of capital from the capital rich countries to
  the capital poor countries. To measure development, I use log GDP per capita
  (constant 2005 US\$) from World Development Indicators.

\item GDP growth may be a proxy of potential returns,

\item Labor quality: As one primary factor of production, labor matters greatly
  to firms' productivity and profit. To measure labor quality, I use the average
  years of schooling of adult, taken from the UNDP's Human Development
  Report.\footnote{Since Taiwan is not included in UNDP's and World Bank's data,
    I collected its statistics from the Taiwanese Statistical Website.}

\item Democracy: Democracy has been a mainstay in the political science
  literature on FDI. Scholars have argued that MNCs want to invest in democratic
  regimes for various reasons, including stable policy, credible commitment, and
  strong property rights \citep{Ahlquist2006, Li2003, Jensen2003}. On the other
  hand, recent works have also argued that democratic regimes want FDI more than
  autocratic regimes \citep{Pandya2016}. Thus, it is unclear whether the
  observed high level of FDI in democracies is due to the push or the pull
  factors. By controlling for countries' preference in the two-sided matching
  model, I can better estimate the effect of democracies on firms' utility. I
  measure democracy using the binary Demoracy \& Dictatorship, developed by
  \citet{Cheibub2009b}.
\end{itemize}

\message{ !name(AnhLe_dissertation.tex) !offset(-96) }

\end{document}
