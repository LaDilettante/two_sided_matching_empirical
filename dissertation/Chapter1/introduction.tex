\chapter{Introduction}

Much of our social, economic, and political life is governed by two-sided
matching markets. In these matching markets, actors from two disjoint sets evaluate the
characteristics of someone on the other side and
voluntarily form a match if both deem each other satisfactory.\footnote{Throughout the dissertation, I
  use ``two-sided matching market'' and ``matching market'' interchangeably. On
  the other hand, note that a two-sided market is not necessarily a
  matching market \citep{Rysman2009}.} Marriage is a prominent example
of such matching process. Others include the matching between firms and workers, federal
judges and law clerks, the \textit{formateur} of a coalition government and
other minority parties, or countries and multinational corporations (MNCs) that
are looking for a location to invest.

Two-sided matching market is substantively consequential because it often
involves scarce, indivisible goods, such as life commitment to a marital
partner or political allegiance in a coalition government. It is also
intellectually interesting because the market outcome depends on the actions of
both sides, requiring different game theoretic and empirical approaches from
those used to analyze one-sided markets.

This chapter will proceed as follows. First, I discuss the game theoretic literature of two-sided matching
models, where much of the terminology and insight originate. I will highlight
key results that are relevant to the empirical studies of matching market.
Second, I discuss existing empirical models to estimate actors' preference in matching markets.

\section{Game theoretic models of matching markets}

\citet{Gale1962} was the first to study the matching market. In this paper, the authors
consider the marriage market in which there are two finite and disjoint sets of
actors, men and women. Both men and women evaluate each other's income,
appearance, or background, to consider forming a marriage.

The outcome of this market is a set of marriages, with none of some of people
prefer to remain single. We define such set of marriages a \textit{matching} $\mu$, which
is a one-to-one function
that matches a man with a woman.  We refer to $\mu(x)$ as the \textit{mate} of
$x$. For convenience,
we say that if an individual decides to remain single, they are matched
with themselves.

A matching can be improved in two ways. First, an individual may prefer to
remain single than to be matched with his or her mate $\mu(x)$ under the current
matching $\mu$. Second, a man and a woman may prefer
to be with one another rather than whom they are currently matched with.
We define
that a matching $\mu$ is \textit{stable} if it cannot be improved upon by any individual
or any pair of agents.  

The first key result from the game theory literature is that
for any set of preference, there will always be a stable matching. This means
that it is reasonable for us to assume that the empirical matching we observe in
real matching market is stable, and that the agents' utility cannot be further
improved. Our empirical model of the two-sided matching market thus needs to
describe a process that produces a stable matching.\footnote{\citet{Gale1962}
  prove a stable matching always exists by describing a procedure that produces
  such a stable matching. The procedure works as follows. In the first stage, every man proposes to his preferred mate. Every
  woman rejects all of her suitors except the one that she most prefers.
  However, she does not yet accept her (so far) favorite suitor, but keeps him along. In the
  second stage, every man that was rejected in the previous round proposes to his second choice. Every
  woman then picks her favorite from the set of new proposers and the man she
  keeps along from the previous round. The procedure continues until there is no longer any
woman that is unmatched, at which point women finally accept their current
favorite choices. (This procedure is called \textit{deferred acceptance} to
capture the fact that women defer accepting her favorite choice until
the last round in case better options become available.) The resulting match is stable because, throughout the procedure, every woman has received all
the offers that would have been made to her, and she has chosen her favorite
among all of those offers. If there were any other man that she would prefer to
her current match, that man would not have been available to her. Therefore, the
final match cannot be further improved by any man or woman.}

A central coordinator employing the \textit{deferred acceptance} algorithm is
guaranteed to come up with a stable matching for the market.\footnote{Indeed,
  the deferred acceptance procedure was used in the market for US medical
  residency with enthusiastic participation from medical students and hospitals.
  The high participation rate indicates that the matching produced is stable enough to entice students and
hospitals away from arranging their own matches outside of the centralized
market.} However, in a decentralized market, such as the regular labor market or
the FDI market,
would agents be able to reach this outcome by themselves? The second key result
from the game theory literature is that stable matching in decentralized
matching market is indeed possible. For example, \citet{Roth2016} show that, starting from
an arbitrary matching, the market can converge to a stable matching with
probability 1 if we allow random blocking pairs, i.e. two individuals that are
not matched but prefer each other to their current match, to
break off and form their own match. In addition, \citet{Adachi2003} shows that a random search process, in
which pairs of man and woman randomly meet, evaluate, and decide whether to pair up, will
converge towards a stable matching if the search cost is negligible.\footnote{In
this model, searching has a time cost. Thus negligible search cost is modeled as
agents having a time discount close to 1.} This result further suggests that the
empirical matching we observed is stable, compelling our empirical model to
describe a process that produces a stable matching.

In generalizing the one-to-one matching market (e.g. marriage) to the
many-to-one matching market (e.g. college admission, labor market),
\citet{Roth1992} show that key results remain valid, albeit requiring additional
assumptions. One important assumption is that firms treat workers as substitutes, not
complements. In other words, this means that firms never regret hiring a worker
even if
another worker is no longer available. Our empirical analysis should focus on
many-to-one matching markets where agents have such ``substitutable
preference.'' Otherwise, a stable matching is not guaranteed, agents' utility functions
are interdependent, and it becomes unclear what kind of matching process our
empirical model should approximate.

\section{Empirical models of matching markets}

The game theory literature takes the agents' preference as given and proves the
existence of a stable matching. In contrast, empirical models of matching
markets takes the observed matching as given and attempt to estimate the agents' preference.

Unfortunately, most extant empirical models fail to adequately account for the
structure of a two-sided matching market. Often, researchers simply analyze the
market from one side, e.g. estimating a firm's preference by looking at the type
of workers it hires. This approach does not take into account the fact that, in
matching markets, a match depends not only on the agent's preference but also
his opportunity. For example, a farm may prefer to hire workers with many years
of education but cannot do so because highly-educated workers do not want to
work on farms. Modeling this interaction between preference and opportunity is
the key contribution of this dissertation.

Alternatively, some researchers measure agents' preferences by surveying them
directly \citep{Posner2001, Sprecher1994}. While this approach circumvents the
need to disentangle preference and opportunity, it can only measure agents'
\textit{stated} preference. In addition, such surveys require high effort while data on
final matching (e.g. married couples, workers' current job, country location of
MNCs) are widely available. This dissertation aims to make use of such available
data to estimate agents' \textit{revealed} preference.

Below I discuss existing empirical models of matching markets. First, I discuss two
markets of interest to political scientists: the US federal clerkship market and
the ``market'' for forming a coalition government. Researchers in both subject
areas have not approached the problem with an empirical model that adequately
captures its two-sided dynamics.

Second, I examine models from other disciplines that do take into account the
two-sided dynamics of matching markets. I start with machine learning models
applied to online marketplaces such as AirBnB or dating sites such as eHarmony.
Then, I discuss the statistical models of the labor market \citep{Logan1996} and the marriage
market \citep{Logan2008}, which are most relevant to our goal of estimating agents' preference
based on observed match data. These statistical models serve as the foundation
of my empirical approach.

\subsection{US federal clerkship market}

In the US, graduates at top law schools vie for the best federal
clerkships every year. These temporary, one-to-two-year positions are the launching pad for Supreme Court clerkship,
prestigious teaching jobs, or employment at top law firms. On the other side,
federal judges also compete for the best law graduates, who help reduce the
judges' workload, from copy-editing to drafting the opinions
\citep{Gulati2016, Posner2001}. Because the first clerkship tends to have an ideological influence
on law graduates, this matching market has important implications for the
polarization of the judicial branch \citep{Liptak2007}.

The market for US federal clerkship has been noted as a classic case of a two-sided market.
Clerks look for positions that provide not only prestige and connection but also
comfortable quality of life \citep{Posner2001}. Judges select law graduates
based on not only academic credentials but also, as some argue, ideology, gender,
and race \citep{Slotnick1984}. This market also suffers from strategic behavior
emblematic of a matching market, such as offers being
made aggressively early and with a short time to accept \citep{Posner2001, Posner2007}.

One approach to estimating the preference of agents in this market is surveying clerks and judges
directly \citep{Peppers2008}. However, as discussed, this approach only measures 
stated preference, which is likely to suffer from social desirability bias when it comes to dimensions
that we care about most such as matching based on ideology, gender, or
race.

Other approaches estimate revealed preference by using observed hiring
outcome. However, no existing study has properly taken into
account the two-sided nature of the market, thus confusing the effects of
preference and opportunity. For example, \citet{Bonica2017} use political
contribution data (DIME dataset) to measure political ideology, then correlates
the ideology of the hiring judge and the ideology of his clerks. This
approach does not take into account the pool of applicants, leading to
conclusions such as conservative judges hire more liberal clerks
than conservative clerks \citep[31]{Bonica2017}. This curious finding has a
potentially simple explanation: the pool of top law graduates tend to be
overwhelmingly liberal, leaving conservative judges not much choice.
Despite this issue, the authors proceed to measure judges' ideology by taking
the average of their clerks' ideology. Without taking the pool of applicants into
account, they may wrongly conclude that conservative judges are more liberal than
they actually are.

In another approach, \citet{Rozema2016} model the process as a discrete choice problem, in which clerks are
differentiated products that Supreme Court justices select to
maximize their utilities. Their model does not consider what clerks think about
the offer. \citet{Rozema2016} can make this assumption because they focus on
Supreme Court clerkship, whose unparalleled prestige ensures that any offer made
will be accepted. However, if we want to extend the model to the
broader market of federal clerkship, such assumption is untenable.

\subsection{The market for forming a coalition government}

Besides election, government formation is the most consequential
political process in determining which government people are subject to. Most
extant studies of government formation are either game theoretic models or thick, ``inside-the-Beltway''
narrative of what happened. Potential advances can be made if we consider
government formation as a two-sided matching market, with the \textit{formateur} party
on one side and other minority parties on the other.\footnote{The
  \textit{formateur} party could be the one with the procedural power to set up
  the coalition, e.g. the incumbent party, or the largest party in established
  coalitions.}

An two-sided matching model of government formation would complement the game theory literature that models politicians as policy-seeking (as
opposed to office-seeking) \citep{Laver1998}. When politicians are policy-seeking, they have
positions that can be modeled as their characteristics. Then, parties can be
modeled as choosing one
another to form a coalition based on their policy positions, akin to men and women
choosing one another to form a marriage based on their height or income.\footnote{In
  contrast, when politicians are office seeking, the only coin of the realm is the number of legislative seats that a
party controls. It determines the inclusion of the party in the government, its portfolio
allocation, etc. In this framework, concept like power indices and dominant
parties is all about which coalition parties can join to turn it into a winning /
losing coalition.} As the game theory literature suggests, ideologically compact
coalitions are more valuable because they entail fewer costs in policy
compromises \citep{DeSwaan1973}. With the empirical matching model, we can test if parties do indeed
prefer others that are ideologically close to themselves ideologically.

In addition, an advantage of the two-sided matching approach is a its ability to consider
multidimensional policy space. By considering a party's
positions on various policies as their covariates, we would be able to
estimate parties' relative preference for ideological proximity across policy dimensions.

\subsection{The FDI market}

To be introduced here, or kept until its own empirical chapter?

\subsection{Recommender system for online two-sided markets}

In recent years, the Internet witnesses the proliferation of two-sided
matching markets such as online market place (AirBnB), dating
site (eHarmony), or job board (Elance). To allow their customers to discover a
match quicker, these sites often build a recommender system that make
suggestions to their users.\footnote{To clarify, the term ``recommender system'' typically
  refers to how platform make recommendation to users based on the reviews of
  users like them. This is not what we are discussing here. Even though there is
also a latent preference we want to uncover in that system, there is not a two
sides choosing one another there.} To maximize user engagement and
profitability, these sites are incentivized to
make recommendations that resemble a stable matching so
that users get the best match. To find the stable matching, they have to
first estimate the preferences of their users.

While most of these algorithms are proprietary, some academic
publications do address this problem. An interesting approach is \citet{Tu2014}, which uses the Latent Dirichlet
allocation (LDA) model to uncover the latent
types of the users based on their observed match on an
online dating platform.\footnote{Besides \citet{Tu2014}, \citet{Hitsch2010,
    Goswami2014} are two other attempts to estimate users' preference in online
  matching markets. However, these papers take a simple one-sided approach,
  ignoring the interplay between preference and opportunity. Therefore, I don't
  discuss them further here.} In the original application of LDA model in
topic modeling, each document is a mixture of latent topics, and each
topic is a distribution over words. In this application, each user is a mixture
of latent ``types,'' and each type is a distribution signifying relative
preference over various mates' features. For example, the ``outdoor type'' may have
higher preference for athleticism or dog ownership over other traits.

While the LDA model works well for the online dating market, it is not
applicable to most social science problems for two reasons. First, this model
requires data of users reaching out to multiple partners rather than just the
final match. Second, while the LDA model uncovers users'
latent types, most social scientists want to estimate the preference of
specific, known types (e.g. how different regime types may prefer different
characteristics of an MNC).

\subsection{Two-sided models for the labor and marriage markets}

To be introduced here, or in their own chapters?

\section{Conclusion}

Roadmap for the rest of the dissertation