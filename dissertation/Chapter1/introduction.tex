\chapter{Introduction}

Much of our economic and political life is governed by a two-sided matching
process, in which two sets of actors evaluate each other's characteristics and
voluntary form a match if deemed satisfactory. Marriage is a prominent example
of such process. Others include the matching of firms and workers, federal
judges and law clerks, countries and multinational firms who want to make an
investment, or the \textit{formateur} of a government and other minority parties.

Two sided matching market is substantively consequential because it 
involves scarce, one-of-a-kind ``good'', such as a life commitment to a marital
partner, or a political allegiance in a government formation---certainly not the
commodity like loaves of bread. It is also intellectually interesting because
its different structure means that our understanding of the normal, one-sided
market is inadequate to explain and analyze.

A two-sided market means that the market involves two disjoint set
of actors, and one side decision affect the other side \citep{Rysman2009}  who evaluate each other's characteristics. For example, in a marriage
market, both men and women evaluate one another's income, appearance,
background, to consider a match. This stands in contrast with one-sided market
in which only one side makes the judgment whether a transaction exist, e.g. a
grocery shopper evaluates the plumpness of the grocer's tomato, while the grocer
does not evaluate the shopper. This means that whether a transaction occurs
depend not only on an individual's demand, but also on the individual's
opportunity. Indeed, while we typically assume that in a grocery market, all the
milk and tomato are always available to buy, and the sole decision is to buy how
much, in a marriage market, the no less important question is what is available
to the individual.

A matching market means that once a match is being made, the transaction is
exclusive between the two individuals being matched. Once a match has
been made, both sides are no longer available on the market. This results in
strategic behaviors, i.e. delaying accepting an offer, falsifying preference,
that causes market failures. For example, in many matching markets, such as the
American physician market 1945-51, the American law graduate market, the elite
Japanese university graduate markets, all suffer from the problem of offers
being made aggressively earlier in order to scoop the best candidates from the market.

The study of two sided matching markets start within the market design subfield
of economics, where scholars are chiefly concerned with explaining the market
failures in the market, and how to design a mechanism to resolve such failure.
In such studies, the preferences of the actors are given, and the goal is to
find a ``stable'' matching. 

The first work in this tradition is Gale and Shapely (1962) and Shapely and
Shubik (1972). Some key proofs:
- From any set of preference, a stable matching always exist
- From a random matching, we will converge to a stable matching with probability one.

\subsection{History of the game theory}

\citep{Gale1962} was the first to study the one-to-one matching market, named
the marriage market. In such a market, there are two finite and disjoint set of
actors, men and women. Each man has preference over each woman, and vice versa.
To each man and woman the option of remaining single is always available.

A matching $\mu$ is a function that matches a man with a woman. For convenience,
we can say that if a man or woman decides to remain single, they are matched
with themselves. We refer to $\mu(x)$ as the \textit{mate} of $x$.

A matching can be improved upon in two ways. First, an individual can prefer to
remain single than to be matched with his or her mate $\mu(x)$ under the current
matching $\mu$. Second, a man and a woman can improve a matching if they prefer
to be with one another rather than whom they are currently matched with.
Therefore, we define
that a matching $\mu$ is stable if it cannot be improved upon by any individual
or any pair of agent.  

The first result from \citep{Gale1962} that is relevant to our purpose is that
for any set of preference, there will always be a stable matching. This means
that it is reasonable for us to assume that the process that generates the final
match we observe in the data is one of agents maximizing their utilities.
Crucially, this provides the justification for us modeling their choice using the familiar tools in the
discrete choice literature. 

When generalize this model to many-to-one matching setting of firm and workers,
all the striking results remain, but we do have to make the assumption of
substitutability, meaning that firms treat workers as substitute, not
complements. Formally, this means that firm never regret hiring a worker if
another worker does not accept the offer. This is certainly not universally
true: a football team strategizing for passing play may want to hire both a good
passer and a good running back, and in the case one does not accept, will go to
another strategy altogether.

\subsection{Current way to measure preference}

The marriage market

The law graduate markets

The FDI market

\subsection{The two-sided matching approach}

The labor market

The marriage market

The machine learning approach