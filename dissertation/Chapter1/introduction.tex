\chapter{Introduction}

Much of our economic and political life is governed by a two-sided matching
process, in which two sets of actors evaluate each other's characteristics and
voluntary form a match if deemed satisfactory. Marriage is a prominent example
of such process. Others include the matching of firms and workers, federal
judges and law clerks, countries and multinational firms who want to make an
investment, or the \textit{formateur} of a government and other minority parties.

Two sided matching market is substantively consequential because it 
involves scarce, one-of-a-kind ``good'', such as a life commitment to a marital
partner, or a political allegiance in a government formation---certainly not the
commodity like loaves of bread. It is also intellectually interesting because
its different structure means that our understanding of the normal, one-sided
market is inadequate to explain and analyze.

A two-sided market means that the market involves two disjoint set
of actors, and one side decision affect the other side \citep{Rysman2009}  who evaluate each other's characteristics. For example, in a marriage
market, both men and women evaluate one another's income, appearance,
background, to consider a match. This stands in contrast with one-sided market
in which only one side makes the judgment whether a transaction exist, e.g. a
grocery shopper evaluates the plumpness of the grocer's tomato, while the grocer
does not evaluate the shopper. This means that whether a transaction occurs
depend not only on an individual's demand, but also on the individual's
opportunity. Indeed, while we typically assume that in a grocery market, all the
milk and tomato are always available to buy, and the sole decision is to buy how
much, in a marriage market, the no less important question is what is available
to the individual.

A matching market means that once a match is being made, the transaction is
exclusive between the two individuals being matched. Once a match has
been made, both sides are no longer available on the market. This results in
strategic behaviors, i.e. delaying accepting an offer, falsifying preference,
that causes market failures. For example, in many matching markets, such as the
American physician market 1945-51, the American law graduate market, the elite
Japanese university graduate markets, all suffer from the problem of offers
being made aggressively earlier in order to scoop the best candidates from the market.

The study of two sided matching markets start within the market design subfield
of economics, where scholars are chiefly concerned with explaining the market
failures in the market, and how to design a mechanism to resolve such failure.
In such studies, the preferences of the actors are given, and the goal is to
find a ``stable'' matching. 

The first work in this tradition is Gale and Shapely (1962) and Shapely and
Shubik (1972). Some key proofs:
- From any set of preference, a stable matching always exist
- From a random matching, we will converge to a stable matching with probability one.

\subsection{History of the game theory}

\citep{Gale1962} was the first to study the one-to-one matching market, named
the marriage market. In such a market, there are two finite and disjoint set of
actors, men and women. Each man has preference over each woman, and vice versa.
To each man and woman the option of remaining single is always available.

A matching $\mu$ is a function that matches a man with a woman. For convenience,
we can say that if a man or woman decides to remain single, they are matched
with themselves. We refer to $\mu(x)$ as the \textit{mate} of $x$.

A matching can be improved upon in two ways. First, an individual can prefer to
remain single than to be matched with his or her mate $\mu(x)$ under the current
matching $\mu$. Second, a man and a woman can improve a matching if they prefer
to be with one another rather than whom they are currently matched with.
Therefore, we define
that a matching $\mu$ is stable if it cannot be improved upon by any individual
or any pair of agent.  

The first result from \citep{Gale1962} that is relevant to our purpose is that
for any set of preference, there will always be a stable matching. This means
that it is reasonable for us to assume that the process that generates the final
match we observe in the data is one of agents maximizing their utilities.
Crucially, this provides the justification for us modeling their choice using the familiar tools in the
discrete choice literature.\footnote{\citep{Gale1962} provides a constructive
  proof of stable matching. They describe the process, called
  \textit{deferred acceptance}, that produces the stable matching. The process
  works as follows. In the first stage, every man proposes to his preferred mate. Every
  woman rejects all of her suitors except the one that she most prefers.
  However, she does not yet accept her favorite suitor (so far), but keep him along. In the
  second stage, every man that was rejected proposes to his second choice. Every
  woman then picks her favorite from the set of new proposers and the man she
  keeps along from the last round. The procedure continues until there is no longer any
woman that is unmatched. The resulting match is stable because, throughout the process, every woman has received all
the offers that would have been made to her, and she has chosen her favorite
among all of those offers. If there is any other man that she would prefer to
her current match, that man would not be available to her. Therefore, the final match cannot be further improved
by any man or woman.}

\citep{Gale1962} treats the case of many-to-one matching (which they call the
``college admission'' game) essentially the same. However, \cite{Roth1992} shows
that a well defined many-to-one matching game is slightly different. However, hen generalize this model to many-to-one matching setting of firm and workers,
all the striking results remain, but we do have to make the assumption of
substitutability, meaning that firms treat workers as substitute, not
complements. Formally, this means that firm never regret hiring a worker if
another worker does not accept the offer. This is certainly not universally
true: a football team strategizing for passing play may want to hire both a good
passer and a good running back, and in the case one does not accept, will go to
another strategy altogether.

So we know that a stable match always exist, meaning it is possible for agents
to reach a final match that maximize their utilities. A central coordinator
employing the \textit{deferred acceptance} algorithm is guaranteed to reach this
stable match. However, in a decentralized market without a central coordinator,
would agents be able to reach this outcome by themselves? How likely is that
this will happen. This matters for our empirical approach because we need to
know whether the final outcome we observe is indeed a stable match, i.e. which
would inform our estimation strategy.\citep{Roth2016} shows that, starting from
an arbitrary matching, the matching can converge to a stable matching with
probability 1 if we allow random blocking pairs, i.e. a woman and a man that are
not matched to one another but prefer each other to their current match, to
match. In addition, \citep{Adachi2003} shows that a random search process, in
which man and woman randomly meet, evaluate, and decide to pair or not, will
converge towards a stable match assuming that the search cost is negligible
(specified as having a time discount = 1). This gives us confidence that the
data we observed is close to a stable match, a fact that will guide us in our
modeling strategy.

\subsection{Empirical studies of two-sided matching market}

Given the importance of many two sided matching markets, there have been many
empirical studies that try to estimate the preference of participants in these
markets. Below I discuss several examples, starting with the marriage and the
labor market, where the two-sided matching literature started. Then I discuss
the law graduate market and the FDI market, two markets that are relevant to
politics, whose two-sided nature has not been fully appreciated and modeled
accordingly.

Estimating preference is a difficult task, and throughout the discussion it will
become apparent that most empirical approaches either 1) use survey to get the
stated preference, not revealed preference, or 2) failing to consider the
two-sided nature of the market in their estimation method, thus unable to
distinguish between preference and opportunity. Our two-sided matching model
aims to address both these shortcomings: it will estimate preference based on
observed match, i.e. revealed preference, and will be able to distinguish
between the effect of opportunity and choice.

\subsubsection{Labor market}

\citep{Abowd1999} is also estimating preference, disentangling the two-sided
effect, but require salary data.

\subsubsection{Law graduate market}

In the United States, graduates at top law schools vie for the best federal
clerkship. These temporary, one-to-two-year position are the launching pad for Supreme Court clerkship,
prestigious teaching jobs or positions at top law firms. On the other hand,
federal judges also compete for the best law graduates, who help reduce the
judges' workload, from copy-editing, to research, and even drafting the opinions
\citep[795]{Gulati2016, Posner2001}

\citep{Rozema2016} is the first quantitative model of the law clerk market. It's
the first to model the process as a discrete choice problem, in which clerks are
differentiated products that the Supreme Court justices pick in order to
maximize their utilities. However, it doesn't consider the other side of the
market, i.e. the decision of the clerks. This article gets away with this
assumption because a Supreme Court clerkship is the most prestigious legal
position for law graduates, so the authors argue that no graduates will turn
down an offered clerkship. However, if we want to extend the model to the
broader market for law clerks, it is imperative to consider the utility of the
law clerks as well.

Indeed, as \citep{Posner2001, Posner2007} surveyed, the competition among
federal judges for top law graduates is fierce, causing the starting date for
making offer to law graduates to be continually pushed up. Law graduates not
only care about the prestige of the job, but also other factors of a job like
quality of life. A judge confessed that, living in a country town, he feels
pressured to start making offer earlier to have any chance of recruiting a good
law graduates \citep{Posner2001}. The competition is so fierce that, despite a moratorium on making
offer earlier than the 3rd year of the law graduates, most judges say that their
peers mostly don't adhere to the moratorium \citep{Posner2007}.

\citep{Bonica2017} is another example of how empirical study that doesn't take
into the two sided nature of the market will produce curious empirical study.
For example, they argues that Judges hire clerks with consistent ideology (p
21), and that they tend to hire clerks with the same ideology as theirs.
However, they also note that conservative tend to hire more liberal clerks, more
so than liberal clerks hiring conservative clerk. This potentially is because
most top school clerks are liberal. Without taking the pool of applicant into
account, we can wrongly mistake this as conservative judges having a preference
for liberal clerks. Indeed, they calculate clerk-based ideology for judge by
simply taking the average (p 35). This will be wrong because what a true
clerk-based measure of ideology would be the judge's preference for the clerk's
ideology, which I can measure here.

\subsubsection{Government formation}



\subsubsection{The FDI market}

\subsection{The two-sided matching approach}

The labor market

The marriage market

The machine learning approach