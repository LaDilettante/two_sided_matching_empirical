\chapter{Introduction}

Much of our economic and political life is governed by a two-sided matching
process, in which two sets of actors evaluate each other's characteristics and
voluntary form a match if deemed satisfactory. Marriage is a prominent example
of such process. Others include the matching of firms and workers, federal
judges and law clerks, countries and multinational firms who want to make an
investment, or the \textit{formateur} of a government and other minority parties.

Two sided matching market is substantively consequential because it 
involves scarce, one-of-a-kind ``good'', such as a life commitment to a marital
partner, or a political allegiance in a government formation---certainly not the
commodity like loaves of bread. It is also intellectually interesting because
its different structure means that our understanding of the normal, one-sided
market is inadequate to explain and analyze.

A two-sided market means that the market involves two disjoint set
of actors, and one side decision affect the other side \citep{Rysman2009}  who evaluate each other's characteristics. For example, in a marriage
market, both men and women evaluate one another's income, appearance,
background, to consider a match. This stands in contrast with one-sided market
in which only one side makes the judgment whether a transaction exist, e.g. a
grocery shopper evaluates the plumpness of the grocer's tomato, while the grocer
does not evaluate the shopper. This means that whether a transaction occurs
depend not only on an individual's demand, but also on the individual's
opportunity. Indeed, while we typically assume that in a grocery market, all the
milk and tomato are always available to buy, and the sole decision is to buy how
much, in a marriage market, the no less important question is what is available
to the individual.

A matching market means that once a match is being made, the transaction is
exclusive between the two individuals being matched. Once a match has
been made, both sides are no longer available on the market. This results in
strategic behaviors, i.e. delaying accepting an offer, falsifying preference,
that causes market failures. For example, in many matching markets, such as the
American physician market 1945-51, the American law graduate market, the elite
Japanese university graduate markets, all suffer from the problem of offers
being made aggressively earlier in order to scoop the best candidates from the market.

The study of two sided matching markets start within the market design subfield
of economics, where scholars are chiefly concerned with explaining the market
failures in the market, and how to design a mechanism to resolve such failure.
In such studies, the preferences of the actors are given, and the goal is to
find a ``stable'' matching. 

The first work in this tradition is Gale and Shapely (1962) and Shapely and
Shubik (1972). Some key proofs:
- From any set of preference, a stable matching always exist
- From a random matching, we will converge to a stable matching with probability one.

\subsection{History of the game theory}

\citep{Gale1962} was the first to study the one-to-one matching market, named
the marriage market. In such a market, there are two finite and disjoint set of
actors, men and women. Each man has preference over each woman, and vice versa.
To each man and woman the option of remaining single is always available.

A matching $\mu$ is a function that matches a man with a woman. For convenience,
we can say that if a man or woman decides to remain single, they are matched
with themselves. We refer to $\mu(x)$ as the \textit{mate} of $x$.

A matching can be improved upon in two ways. First, an individual can prefer to
remain single than to be matched with his or her mate $\mu(x)$ under the current
matching $\mu$. Second, a man and a woman can improve a matching if they prefer
to be with one another rather than whom they are currently matched with.
Therefore, we define
that a matching $\mu$ is stable if it cannot be improved upon by any individual
or any pair of agent.  

The first result from \citep{Gale1962} that is relevant to our purpose is that
for any set of preference, there will always be a stable matching. This means
that it is reasonable for us to assume that the process that generates the final
match we observe in the data is one of agents maximizing their utilities.
Crucially, this provides the justification for us modeling their choice using the familiar tools in the
discrete choice literature.\footnote{\citep{Gale1962} provides a constructive
  proof of stable matching. They describe the process, called
  \textit{deferred acceptance}, that produces the stable matching. The process
  works as follows. In the first stage, every man proposes to his preferred mate. Every
  woman rejects all of her suitors except the one that she most prefers.
  However, she does not yet accept her favorite suitor (so far), but keep him along. In the
  second stage, every man that was rejected proposes to his second choice. Every
  woman then picks her favorite from the set of new proposers and the man she
  keeps along from the last round. The procedure continues until there is no longer any
woman that is unmatched. The resulting match is stable because, throughout the process, every woman has received all
the offers that would have been made to her, and she has chosen her favorite
among all of those offers. If there is any other man that she would prefer to
her current match, that man would not be available to her. Therefore, the final match cannot be further improved
by any man or woman.}

\citep{Gale1962} treats the case of many-to-one matching (which they call the
``college admission'' game) essentially the same. However, \cite{Roth1992} shows
that a well defined many-to-one matching game is slightly different. However, hen generalize this model to many-to-one matching setting of firm and workers,
all the striking results remain, but we do have to make the assumption of
substitutability, meaning that firms treat workers as substitute, not
complements. Formally, this means that firm never regret hiring a worker if
another worker does not accept the offer. This is certainly not universally
true: a football team strategizing for passing play may want to hire both a good
passer and a good running back, and in the case one does not accept, will go to
another strategy altogether.

So we know that a stable match always exist, meaning it is possible for agents
to reach a final match that maximize their utilities. A central coordinator
employing the \textit{deferred acceptance} algorithm is guaranteed to reach this
stable match. However, in a decentralized market without a central coordinator,
would agents be able to reach this outcome by themselves? How likely is that
this will happen. This matters for our empirical approach because we need to
know whether the final outcome we observe is indeed a stable match, i.e. which
would inform our estimation strategy.\citep{Roth2016} shows that, starting from
an arbitrary matching, the matching can converge to a stable matching with
probability 1 if we allow random blocking pairs, i.e. a woman and a man that are
not matched to one another but prefer each other to their current match, to
match. In addition, \citep{Adachi2003} shows that a random search process, in
which man and woman randomly meet, evaluate, and decide to pair or not, will
converge towards a stable match assuming that the search cost is negligible
(specified as having a time discount = 1). This gives us confidence that the
data we observed is close to a stable match, a fact that will guide us in our
modeling strategy.

\subsection{Empirical studies of two-sided matching market}

Given the importance of many two sided matching markets, there have been many
empirical studies that try to estimate the preference of participants in these
markets. Below I discuss several examples, starting with the marriage and the
labor market, where the two-sided matching literature started. Then I discuss
the law graduate market and the FDI market, two markets that are relevant to
politics, whose two-sided nature has not been fully appreciated and modeled
accordingly.

Estimating preference is a difficult task, and throughout the discussion it will
become apparent that most empirical approaches either 1) use survey to get the
stated preference, not revealed preference, or 2) failing to consider the
two-sided nature of the market in their estimation method, thus unable to
distinguish between preference and opportunity. Our two-sided matching model
aims to address both these shortcomings: it will estimate preference based on
observed match, i.e. revealed preference, and will be able to distinguish
between the effect of opportunity and choice.

Or they have demanding requirement on data, going further than just the simple match.

\subsubsection{Labor market}

\citep{Abowd1999} is also estimating preference, disentangling the two-sided
effect, but require salary data.

\subsubsection{Recommender system for online two-sided markets}

In recent years, many online market place proliferate such as AirBnB, dating
site like eHarmony, or job board like o-lance. To save money, these sites must
make recommendations to their users in a way that resembles a stable match so
that their users are more engaged with the site instead of being dissatisfied
with the match.\footnote{To clarify, the term ``recommender system'' typically
  refers to how platform make recommendation to users based on the reviews of
  users like them. This is not what we are discussing here. Even though there is
also a latent preference we want to uncover in that system, there is not a two
sides choosing one another there.}

Even though most of these algorithms are proprietary, there exists some academic
publication addressing this problem. \citep{Hitsch2010, Goswami2014} are two
attempts using one-sided approach, ignoring any strategic component.

An interesting approach is \citep{Tu2014} which uses the Latent Dirichlet
allocation (LDA) model to uncover thethe latent
types of the users and their preferences based on their observed match on an
online dating platform. In the original application of LDA model on
topic modeling, each document is a mixture of latent topics, and each
topic is a distribution over words. In this application, each user is a mixture
of latent ``types,'' and each type is a distribution signifying relative
preference over mates' feature. For example, the ``outdoor type'' may have
higher preference for athleticism or dog ownership over other traits of their
mates.

However, these kinds of models only work because they either flat out ignore the
two sided nature, or have access to data of users making multiple actions (i.e.
reaching out to multiple partner) instead of just the final match. However,
considering that these sites typically put a limit on the number of outreach you
can have in a given time, or considering the simple time cost of making an
outreach, we cannot ignore the strategic component of an user only reaching out
to users that they think they have a chance with. So the preference they
estimate is already confounded by the type of opportunity available. While these
models have decent predictive performance when the pool of applicants are
static, they cannot predict well the change in matches when there is a change in
the opportunity pool.

\subsubsection{Law graduate market}

In the United States, graduates at top law schools vie for the best federal
clerkship. These temporary, one-to-two-year position are the launching pad for Supreme Court clerkship,
prestigious teaching jobs or positions at top law firms. On the other hand,
federal judges also compete for the best law graduates, who help reduce the
judges' workload, from copy-editing, to research, and even drafting the opinions
\citep[795]{Gulati2016, Posner2001}. As clerkship having an outsized influence
on law graduates, studying the law market has important implication for the
polarization of the judicial branch.

This market has long been recognized as a classic case of a two-sided market.
Clerks look for positions with not only prestige and connection but also
comfortable living situation \citep{Posner2001}. Judges select law graduates
based on not only academic credentials but also, some afraid, ideology, gender,
and race \citep{Slotnick1984}. This market also suffers from strategic behavior
emblematic of a matching market, such as offers being
made as early as two years before the clerkship start date, and with a short time to accept \citep{Posner2001, Posner2007}.

The two sided nature of the market makes it difficult to quantitatively study the preference
of the market participants. One approach has been to survey clerks and judges
directly \citep{Peppers2008}. However, this only allows researchers to measure
stated preference, which is unlikely to be accurate when it comes to dimensions
that we care about most such as discrimination based on ideology, gender, or
race.

Other quantitative approaches circumvent this problem by using observed hiring
outcome to study preference. However, no study so far has properly taken into
account the two-sided nature of the market, thus confusing the effects of
opportunity and preference. For example, \citet{Bonica2017} use political
contribution data (DIME dataset) to measure political ideology, then find the
correlation between the hiring judges' ideology and their clerks' ideology. This
approach does not take into account the pool of clerk applicants, which leads to
curious conclusion such as that conservative judges hire more liberal clerks
than conservative clerks \citep[31]{Bonica2017}. This curious finding has a
potentially simple explanation: the pool of top law graduates tend to be
overwhelmingly liberal, and conservative judges may not have much choice.
Despite this drawback, the authors proceed to measure judges' ideology by taking
the average of their clerks' ideology. Without taking the pool of applicant into
account, we may wrongly conclude that conservative judges are more liberal than
they are.

Similarly, \citet{Rozema2016} models the process as a discrete choice problem, in which clerks are
differentiated products that the Supreme Court justices pick in order to
maximize their utilities. However, it doesn't consider what clerks think about
the offer. \citet{Rozema2016} can make this assumption because they focus on
Supreme Court clerkship, whose unparalleled prestige ensures that any offer made
will be accepted. However, if we want to extend the model to the
broader market of federal clerkship, such assumption is untenable.

\subsubsection{Government formation}

Besides election, government formation is probably the most consequential
political process in determining the government that people are subject to. Most
extant studies of government formation has either been game theoretic or thick description, ``inside-the-Beltway''
narrative of what happened. Potential advances be made when we consider
government formation as a two-sided matching market, with the \textit{formateur}
on one side and all other minority parties on the other.\footnote{The
  \textit{formateur} party could be the one with the procedural power to set up
  the coalition, e.g. the incumbent party, or the largest party in established
  coalitions.} There are empirical studies of policy-seeking politicians on portfolio
allocation, but it's broad statement like this agragrian party will take the
agriculture ministry if the party is in the coalition and that there is such a party.

An empirical study of government formation as a two-sided matching market complements the game theory literature that model politicians as policy-seeking (as
opposed to office seeking). When politicians are policy-seeking, they have
positions that can be modeled as their characteristics, and parties choosing one
another to form coalition based on their policy positions akin to men and women
choosing one another to form a marriage based on height or income \cite{Laver1998}.\footnote{In
  contrast, when politicians are office seeking, the only coin of the realm is the number of legislative seats that a
party controls. It determines the inclusion of the party in the government, its portfolio
allocation, etc. In this framework, concept like power indices and dominant
parties is all about which coalition parties can join to turn it into a winning /
losing coalition.} As the game theory literature suggests, ideologically compact
coalitions are more valuable because they entail fewer costs in policy
compromises. With the empirical matching model, we can test if parties do indeed
prefer others closer to themselves ideologically.

In addition, the two sided matching approach has the advantage of studying
multidimensional policy space. It works quite naturally by considering a party's
positions on various policies as their many characteristics.

\subsubsection{The FDI market}

Estimating the preference: democracy and FDI attraction, survey, comparing FDI
and equity, type of FDI (greenfield vs brownfield).

Also demand for the type of FDI.