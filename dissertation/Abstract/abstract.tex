\abstract

A foreign direct investment (FDI) project can only materialize with the consent of both the multinational corporation (MNC) and the host country. However, the literature on FDI has focused only on the preference of MNCs, assuming that all countries are eager to receive FDI. Through various case studies, I show that countries have varied and strategic preference, playing a substantial role in determining where FDI locates. Failing to recognize this two-sided matching nature of the FDI market, not only do existing models of FDI produce wrong estimates of MNCs' preference, they also prevent us from investigating countries' preference. I introduce the two-sided matching model, investigate properties of the model, and apply it to study Japanese FDI in Southeast Asia. I show how to estimate the preference of both MNCs and countries for one another, modeling the two-sided matching process behind FDI location that scholars have always known but never been able to study quantitatively. With this model, scholars can better understand of what drives FDI location and policy makers can better simulate FDI movement under hypothetical policy changes.