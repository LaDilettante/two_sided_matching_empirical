\abstract

A foreign direct investment (FDI) project can only materialize with the consent of both the multinational corporation (MNC) and the host country. However, despite this two-sided nature of the FDI market, the literature on FDI has focused only on the preference of MNCs, assuming that all countries are eager to receive FDI. Through various case studies, I show that countries have varied and strategic preference, playing a substantial role in determining where FDI locates. Failing to recognize this two-sided matching nature of the FDI market, not only do existing models of FDI produce wrong estimates of MNCs' preference, they also prevent us from understanding countries' FDI policies. Therefore, I introduce the two-sided matching model as a more appropriate approach to the FDI market and the Bayesian Markov chain Monte Carlo algorithm to estimate it. Apply the model to study Japanese FDI in Southeast Asia, I show how to estimate the preference of MNCs and countries for one another. With this model, scholars can better understand what drives FDI location, and policy makers can better simulate FDI movement under hypothetical policy changes.