\chapter{Duke Dissertation Class Documentation}
\label{chap:guide}

\cite{Haario2001}

{\verb dukedissertation.cls } --- a document class for dissertations and theses
conforming to the 2011 Duke University guidelines.  This class is by
Michael Gratton and modified by Hugh Crumley.  It is based on the 2004 \LaTeX2$\epsilon$ version of
report.cls and code in the older dukethesis.cls dating back to 1987.

The report.cls is Copyright 1993 1994 1995 1996 1997 1998 1999 2000 2001 2002
2003 2004 The LaTeX3 Project.

The original dukethesis.cls contained work by Mark Holliday, Charlie
Martin, Russ Tuck, Sean O'Connell, Michael Todd, Syam Gadde, and Rajiv
Wickremesinghe.  Some of this work has been folded into the new document class.

This file may be distributed and/or modified under the
conditions of the LaTeX Project Public License, either version 1.3
of this license or (at your option) any later version.
The latest version of this license is in
\url{http://www.latex-project.org/lppl.txt}
and version 1.3 or later is part of all distributions of LaTeX
version 2003/12/01 or later.

This is the June 13,2011 version, 0.5,(updated by Hugh Crumley) by Michael Gratton. \footnote{
E-mail: \href{mailto:mgratton@math.duke.edu}{mgratton@math.duke.edu}}

\section{Features}

This class conforms to the 2011 style guidelines for dissertations,
including:
\begin{enumerate}
\item Page numbers centered in the footer of each page
\item Margins: 1in top, 1in right, 1.5in left, 1in below footer
\item Title signature page, UMI abstract title signature page, and copyright
   page automatically generated at {\verb \maketitle }
\item 'Double' spacing throughout body text (really about 10pt extra instead
   of 12pt extra.)
\item Double spacing between and single spacing within the Table of Contents, List of Tables, List of
   Figures, Bibliography, and in chapter, section titles, and figure/table
   captions.
\item Footnotes are numbered consecutively within a chapter and placed at the
   bottom of the page on which the reference number appears.
\item Page ordering and numbering: roman numeral page numbers appear
   in the frontmatter (prior to the introduction or Chapter 1).  The first
   numbered page is the Abstract (iv).  Arabic numbering from '1'
   starts in the Introduction or Chapter 1 if there's no Introduction.
\item Optional material supported:
	\begin{itemize}
   \item Dedication
   \item Acknowledgements
   \item Introduction (different from 'Chapter 1: Introduction')
   \item Appendices
	\end{itemize}
\end{enumerate}

This class also provides some handy features:
\begin{enumerate}
\item Use the option 'economy' to get a single-spaced document appropriate for
   giving to colleagues.
\item Change your copyright from 'All rights reserved' if you're not actually
   reserving all your rights.
\item New Look: boldface mostly removed in headers for a lighter feel.
   The word 'Chapter' no longer appears on opener pages, only the number.
\end{enumerate}

\section{Limitations}
In it's current form, this class does not support committees larger than
six members, or titles longer than four
lines.  The
figure-to-caption space has been abbreviated, as most plotting programs
provide ample bottom margins.  This default may not be acceptable in all
cases.

\section{Class options}

The class supports the following options.  Options appear in pairs with the
default option in each pair listed first.  The exception is the first listed
option, which merely activates several other options for convenience.
\begin{description}
	\item[economy] Macro.  Enables the options singlespace,
		nogradschool, and nobind. \textbf{Changed in version 0.3}
	\item[gradschool] Default. Produces signature lines and a UMI abstract title page.
	\item[nogragschool] Supresses the above.
	\item[PhD] Default. Format is suitable for a Ph.D dissertation.
	\item[MS] Modifies the format for Masters Theses.  Changes the text on
	the title page, omits the UMI page, and generates warnings when forbidden
	document parts are used (i.e., Biography).
	\item[openany] Default. Allows a chapter to start on any page.
	\item[openright] Chapters only start on right-hand pages.  Only makes
	sense for twosided documents.
	\item[oneside] Default.  Wide margin (where the binding will be) always
		occurs on the left edge of a page.
	\item[twoside] Wide margin occurs on the left of odd pages and the right
		of even pages.  This is for binding duplex printed documents.
	\item[final] Default. No extra marks, include all pictures.
	\item[draft] Prints black bars on pages where the contents overflow the
	margins.  Suppresses the inclusion of graphics for speed.
	\item[doublespace] Default.  Double-spaces body text and adds extra space
		between entries in the Table of Contents, List of Figures, List of
		Tables, List of Abbreviations, and Bibliography.
	\item[singlespace] Normal distances between baselines in all cases.
		Normal spacing in all list-type environments.
	\item[newstyle] Default.  Lighter look for headings.
	\item[oldstyle] Headings in the classic \LaTeX style. \textbf{New in
		version 0.3}
	\item[bind] Default. 1.5in margin for bidning appears on spine-side of
		a page.
	\item[nobind] Left and right margins are both 1.25in. \textbf{New in
		version 0.3}
\end{description}
The default options are chosen so that the document will pass the Ph.D
format specifications of the graduate school.

Here are some handy examples. Format required by graduate school:
\begin{verbatim}
\documentclass{dukedissertation}
\end{verbatim}
Easy-to-read format for printing, sending to collaborators, etc:
\begin{verbatim}
\documentclass[economy]{dukedissertation}
\end{verbatim}
 Suitable for spiral bound copies and the like
\begin{verbatim}
\documentclass[economy, twoside, bind]{dukedissertation}
\end{verbatim}


