\chapter{Two-Sided Matching Model}
\label{chap:model}

Here I present a behavioral model of the two-sided matching market, focusing on
the case of many-to-one matching, proposed by \cite{Logan1996}. For easier
exposition, throughout the chapter I will use the example of the labor market,
where many workers can be matched to one firm.

We assume that the matching process in the labor market happens in two stages.
In the first stage, each firm evaluates each worker in the sample, then decides
whether to hire that worker or not. At the end of this stage, each worker will
have received a set of offers from firms, which we call his \textit{opportunity
  set}. In the second stage, each worker evaluates the firms in his opportunity
set and chooses the firm that he likes best. This constitutes the final,
observed match between a worker and a firm. This is a many-to-one matching
problem because a firm can make offers to multiple workers, none, some, or all
of which can be accepted by workers.

If we assume that firms and workers are utility-maximizing agents, at the end of
this process, no firm or worker would voluntarily change their final matches. As
discussed in Section~\ref{sec:game_theory}, this property is called
\textit{stability} in the game theoretic two-sided matching literature. We want
our model to have this property because matching markets tend to produce stable
matching. Indeed, \citet{Roth1992} show that for any given set of preferences, a
stable match always exist. Furthermore, \citet{Roth2016} and \citet{Adachi2003}
show that a decentralized market with agents making independent,
utility-maximizing decisions can also reach a stable match by itself.

This stability property does not implies that the matches will never change.
Indeed, if actors' preference shift, their characteristics change, or new actors
enter the market, the matches will also change as a result of actors'
recalculating their utility and adjust their decisions. Therefore, since we are
estimating actors' preference using only a snapshot of matching market, we are
making the assumption that on a systemic level, the average characteristics of
the actors and their preference remain sufficiently static for our estimates to
be meaningful.

This chapter will proceed as follows. First, I discuss the utility model for how
firms make offers to workers. Second, I discuss the utility model for how
workers choose the best offer among those extended by firms. Third, I show why
estimating the model with MLE is difficult, and how we can use Bayesian MCMC
approach for estimation. Fourth, I analyze US labor data and demonstrate how to
interpret the model's result.

\section{Modeling firms' decision making}

A firm $j$'s decision on whether to hire worker $i$ rests on two utility
functions. First, firm $j$'s utility for hiring worker $i$ is:

\begin{align}
  U_j(i) &= \bm{\beta}_j' X_i + \epsilon_{1ij}
\end{align}

where $\beta_j$ is a vector of firm $j$'s preference for worker characteristics,
$x_i$ is a vector of worker $i$'s measured values on those characteristics, and
$\epsilon_{1ij}$ is the unobserved component that influences firm $j$'s utility.

On the other hand, the utility of not hiring worker $i$ is:

\begin{align}
  U_j(\neg i) &= b_j + \epsilon_{0ij}
\end{align}

where $b_j$ is the baseline utility of firm $j$, and $\epsilon_{0ij}$ is the
unobserved component that influences firm $j$'s utility.

Firm $j$ will make an offer to hire worker $i$ if $U_j(i) > U_j(\neg i)$.
Relevant worker characteristics (i.e. $X_i$) that a firm may consider are age,
education, or experience. The corresponding $\beta$'s represent the firm's
preference for these characteristics.

This model makes two important assumptions about firms' hiring process. First,
whether a firm decides to hire worker A depends on the characteristics of worker
A alone, and it will continue to hire worker A even if worker B is no longer
available. In other words, firms regard workers as substitutes rather than
complements.\footnote{In the terminology of \citet{Roth1992}, firms are assumed
  to have ``substitutable preference,'' or firms' preference is assumed to have
  the property of substitutability. As discussed in
  Section~\ref{sec:game_theory}, this assumption is necessary to prove the
  existence of stable matching in the case of many-to-one matching.} This
assumption is not universally true. A Hollywood producer may want to hire two
specific lead actors for their chemistry, and if one is unavailable, the other
also has to be replaced. However, for large firms where workers are closer to
swappable cogs than unique superstars, this assumption is reasonable.

Second, the model assumes that the utility of hiring a worker does not depend on
how many other workers accept the offer. In other words, the firm is large
enough to employ all the workers to whom it extends offer without feeling the
effect of diminishing marginal productivity of labor. This assumption is less
restrictive than it may seem. Indeed, we can model the fact that the workers
under consideration are less productive than the previous batch of workers by
allowing firm $j$ to have a high baseline utility $b_j$. Therefore, this
assumption does not require that there is never any diminishing marginal
productivity of labor, only that there is negligible diminishing effect between
the first and the last of the workers under consideration. This assumption is a
reasonable approximation if the firm's labor force is large compared to the
number of workers being considered.\footnote{While not concerned with
  diminishing marginal productivity, \citet{Roth1992} also assume that firms'
  quota, i.e. the number of workers they can accept, is sufficiently large to
  hire everyone in the set of workers under consideration. This assumption
  simplifies the proof that a stable match always exists in the case of
  many-to-one matching.}

In addition to two above assumptions about the process of firm's decision
making, we make three parametric assumptions that are standard in the discrete
choice literature. First, we assume a linear utility function. Second, we assume
that the error terms $\epsilon_{1ij}, \epsilon_{0ij}$ are uncorrelated with one
another and across firms. Third, we assume that the as error terms
$\epsilon_{1ij}, \epsilon_{0ij}$ follow the Gumbel distribution.\footnote{The
  Gumbel distribution is very similar to the normal, only with a slightly fatter
  tail that allows for slightly more extreme variation in the unobserved
  utility. Its density function is $\exp^{-(x + \exp^{-x})}$, with mode 0, mean
  0.5772, and fixed variance $\frac{\pi^2}{6}$. In practice, the difference
  between using Gumbel and independent normal error terms is small
  \citep{Train2009}.} The choice of the Gumbel distribution is largely motivated
by convenience since it allows us to derive the probability of firm $j$ making
an offer to worker $i$ as the familiar binomial logit form:

\begin{align}
  Pr(o_{ij} = 1) &= Pr(U_j(i) > U_j(\neg i)) \\
                 &= Pr(\epsilon_{0ij} - \epsilon_{1ij} <  \bm{\beta}_j ' X_i - b_j) \\
                 &= \frac{\exp({\bm{\beta}_j'X_i})}{1 + \exp({\bm{\beta}_j'X_i})} \label{eq:prob_offer_ij}
\end{align}

The term $b_j$ is absorbed into $\bm{\beta}$ when we add an intercept term to
the covariate matrix $X$.

Once firms have made their offers, each worker $i$ will have a set of offers
from which to pick her favorite. We call this set of offers the
\textit{opportunity set} of worker $i$, denoted $O_i$. Since unemployment is
always an available option, every opportunity set includes unemployment as an
``offer.''\footnote{In our model setup, firms and workers decide sequentially,
  with firms making offers first in order for workers to have opportunity sets
  to choose from. While firms and workers in real life certainly do not act in
  this sequential manner, the idea of the opportunity set is still applicable.
  Workers in the real labor market may not know their exact set of offers, but
  they can certainly guess which firms are within their reach based on their
  characteristics and on guesses about firms' preference.}

The probability of worker $i$ obtaining the opportunity set $O_i$ is:

\begin{align}
  p(O_i | \bm{\beta}) &= \prod_{j \in O_i} p(o_{ij} = 1 | \bm{\beta}) \prod_{j \notin O_i} p(o_{ij} = 0 | \bm{\beta}) \\
                      &= \prod_{j \in O_i} \frac{\exp(\bm{\beta_j} ' X_i)}{1 + \exp(\bm{\beta_j}' X_i)}
                        \prod_{j \notin O_i} \frac{1}{1 + \exp(\bm{\beta_j}' X_i)} \label{eq:conditional_probability_of_offer}
\end{align}

\section{Modeling workers' decision making}

Worker $i$'s utility for the accepting an offer from firm $j$ is:

\begin{align}
  V_i(j) &= \alpha' W_{j} + v_{ij}
\end{align}

where $\alpha$ is a vector of workers' preference for relevant characteristics
of firms, $W_j$ is a vector of firm $j$'s measured values on those
characteristics, and $v_{ij}$ is the unobserved component that influences worker
$i$'s utility.

Worker $i$ evaluates all the firms in her opportunity set and selects the offer
that brings the highest utility. This decision of worker $i$ concludes the
matching process, resulting in the observed final match between a worker and her
chosen firm in our data.

We make two assumptions in modeling the worker's decision making. First, for
simplicity, we assume that workers all firms share the same set of
preferences--hence $\alpha$ does not have a subscript $i$. The model can be
extended so that there is heterogeneous preference among workers, either by
estimating a separate model for each worker type (i.e. no pooling) or by
building a hierarchical model for worker preference (i.e. partial pooling).

Second, we assume that the error term $v_{ij}$ are uncorrelated across $j$. In
other words, the unobserved factors in the utility of one job offer is
uncorrelated to the unobserved factors in the utility of another job
offer.\footnote{This assumption also gives rise to the Independence of the
  Irrelevant Alternatives (IIA) property. IIA implies that the relative odds of
  choosing between two alternatives depend only on the two alternatives under
  consideration. It does not depend on whether other alternatives are available
  or what their characteristics may be. Hence, other alternatives are considered
  ``irrelevant.''} This assumption is most likely not true: if worker $i$ values
some unobserved factors of an offer, she is likely to consider those same
factors in another offer as well. The hope is that we have modeled the observed
portion sufficiently well that the remaining unobserved factors are close to
white noise. In any case, this issue afflicts any application of discrete choice
models and is not unique to our setup.\footnote{The discrete choice literature
  has developed solutions for such correlated error structure, such as nested
  logit, probit, and mixed logit, that can be applied here if we suspect that
  the unobserved portion is strongly correlated.}

Similar to our model of firm's utility, our model of worker's utility has three
additional parametric assumptions that are standard in the literature. First, we
assume that utility is linear. Second, the error term $v_{ij}$ are uncorrelated
across $i$. Third, we model $v_{ij}$ having a Gumbel distribution so that the
probability that worker $i$ will accept the offer of firm $j$ out of all the
offers in its opportunity set $O_i$ takes the conditional logit form
\citep{Cameron2005}:

\begin{align}
  p(A_i = a_i | O_i, \alpha_i) = \frac{\exp(\alpha'W_{a_i})}{\sum\limits_{j:j \in O_i} \exp(\alpha'W_j)} \label{eq:conditional_probability_of_accept}
\end{align}

where $a_i$ is the index of the firm that $i$ accepts to to work for.
Unemployment is indexed as 0.

\section{Model estimation}

Our goal is to estimate the preference of firms and workers, i.e. $\beta_j$ and
$\alpha$. The key insight is that, conditional on the opportunity set being
observed, the model of firms' and workers' decision making is a straightforward
application of the binary logit and conditional logit model. Both models can be
estimated with familiar tools like Maximum Likelihood Estimation (MLE).

However, in most social science research problems, the researcher only observes
the final match $A$ and not the opportunity set $O$. For example, labor market
data typically does not include the set of offers a worker received (or would
have received if she had applied), while data on her current job is widely
available. Similarly in the marriage market or the FDI market, researchers often
do not have the data on the offers being made, and only observe the final
matching between men and women (i.e. who is married to whom) and between MNCs
and countries (i.e. which factory is located where).

\citet{Logan1998}'s solution to this problem is to use the
Expectation-Maximization (EM) algorithm, an iterative method capable of finding
the maximum likelihood estimates when the model depends on unobserved latent
variables (i.e. the unobserved opportunity set in this case)
\citep{Dempster1977}. Our innovation is to estimate the model using a Bayesian
MCMC approach, which offers several advantages. First, our MCMC approach
produces the full posterior distribution, making inference easy. In contrast, EM
only produces point estimates out of the box.\footnote{\citet{Jamshidian2000}
  propose a method for estimating the standard error of EM estimates. However,
  for hypothesis testing, we need further assumptions about the distribution of
  the EM estimates.} Second, our MCMC approach can be faster than EM when the
latent variable, i.e. the opportunity set, is high dimensional
\citep{Ryden2008}.\footnote{Indeed, our opportunity set $O$ is a $(I \times J)$
  matrix of 0s and 1s, where $I$ is the number of workers and $J$ is the number
  of firms. Thus, there are $2^{IJ}$ potential values for the opportunity set,
  which quickly becomes untenable even for a small number of $I$ and $J$. The
  high dimension of $O$ forces \citet{Logan1998} to reduce the data dimension by
  aggregating 17 employers in the data into 5 employer types, e.g. professional
  or blue collar jobs.} Third, within the Bayesian framework, we can naturally
put a hierarchical structure on firms' preference. This allows us to borrow
information across firms, producing more precise estimates even when there is
not a lot of data for a specific firm.

The rest of this section describes how we conduct model estimation.

\subsection{Estimating the model using Metropolis-Hastings}

We are interested in the posterior distribution of workers' and firms'
preference given the observed final match, i.e. $p(\alpha, \beta | A)$.
Unconditioned on the opportunity set, this posterior is difficult to derive or
sample from. Therefore, we instead sample from the augmented posterior
$p(\alpha, \beta, O | A)$, whose density is much simpler to derive.\footnote{See
  \citet{Tanner1987} for a discussion of such data augmentation techniques.}
Specifically,

\begin{align}
  p(\alpha, \beta, O | A) &= \frac{p(A | \alpha, \beta, O) p(\alpha, \beta, O)}{p(A)} \\
                          &\propto p(A|O, \alpha) p(O|\beta) p(\alpha) p(\beta) 
                            \label{eq:posterior_density}
\end{align}

where $p(A|O, \alpha)$ is derived in
\eqref{eq:conditional_probability_of_accept}, $p(O|\beta)$ is derived in
\eqref{eq:conditional_probability_of_offer}, $p(\alpha)$ and $p(\beta)$ are
prior distributions for $\alpha$ and $\beta$. A key insight of this equation is
that the acceptance of offers, i.e. $p(A|O, \alpha)$, depends only on the
opportunity set and on the workers' preference. Similarly, the opportunity sets,
i.e. $p(O|\beta)$, depend only on firms' preference.

Because the opportunity set $O$ is a discrete matrix of 0's and 1's, there is
not any convenient conjugate model for \eqref{eq:posterior_density}, making
Gibbs sampling impossible. Therefore, we use Metropolis-Hastings instead, a
technique to sample from an arbitrary distribution $p(\theta)$ using the
following steps:

\begin{enumerate}
\item Start from an arbitrary value of $\theta$
\item Generate a proposal value $\theta^*$ from the proposal distribution
  $q(\theta^*|\theta)$
\item Calculate the acceptance ratio $MH_{\theta} =
  \frac{p(\theta^*)q(\theta|\theta*)}{p(\theta)q(\theta^*|\theta)}$
\item Accept the proposed value $\theta^*$ with probability $\max(1,
  MH_{\theta})$
\item Repeat step 2-4 until convergence
\end{enumerate}

In our case, we will use symmetric proposal distributions, i.e.
$p(\theta^*|\theta)$ = $p(\theta | \theta^*) \forall \theta, \theta^*$, so that
the MH acceptance ratio simplifies to $MH_{\theta} =
\frac{p(\theta^*)}{p(\theta)}$. In addition, because preference parameters tend
to be correlated, we use an adaptive proposal distribution so that our MCMC
samples have a faster convergence rate \citep{Haario1999, Haario2001}.\footnote{Description
  of the Adaptive Metropolis procedure?}

Below we describe how to sample from the posterior of each parameter in the
model using the Metropolis-Hastings (MH) algorithm. More detailed derivation of
the Metropolis acceptance ratio is included in Appendix~\ref{chap:MH_ratio}. We
ensure that our derivation and implementation of the acceptance ratio is correct
using the unit-testing approach suggested by
\citet{Grosse2014}.\footnote{Describe the unit-testing framework to ensure the
  correctness of MCMC code?}

\subsection{Posterior of the opportunity set $p(O|A, \alpha, \beta)$}

For each worker $i$, we propose a new value $O_i^*$ by flipping random cells in
the current value $O_i$ from 0 to 1 and 1 to 0. Substantively, this is
equivalent to perturbing the opportunity set by randomly making new offers or
withdrawing existing offers. Note that this proposal distribution is indeed
symmetric because proposing $O_i^*$ from $O_i$ and proposing $O_i$ from $O_i^*$
both involve flipping the same cells in the opportunity set. Hence,
$p(O_i^*|O_i) = p(O_i|O_i^*) =$ the probability of selecting these particular
cells out of the opportunity set.

The Metropolis acceptance ratio for the proposed opportunity set $O_i^*$ is

\begin{align}
  MH_O &= \frac{p(O_i^* | A_i, \alpha, \bm{\beta})}{p(O_i | A_i, \alpha, \bm{\beta})} \\
       &= \frac{\sum\limits_{j:j \in O_i} \exp(\alpha'W_j)}{\sum\limits_{j:j \in O_i} \exp(\alpha'W_j) \pm \exp(\alpha' W_{j^*})} \times \exp(\pm \bm{\beta}_{j^*}'X_i)
\end{align}

where $\pm$ evaluates to $+$ if $j^*$ is a new offer being added to the current
opportunity set, and evaluates to $-$ if $j^*$ is an existing offer being withdrawn from
the current opportunity set.

\subsection{Posterior of firms' preference $p(\alpha|A, O, \beta)$}

At the beginning of the MCMC chain, we propose a new $\alpha^*$ using a Normal
proposal distribution centered on the current value $\alpha$ with a hand-tuned
diagonal covariance matrix. Later in the MCMC chain, the covariance matrix of
the proposal distribution is adapted based on past samples to take into account the
correlations across preference parameters \citep{Haario2001}.

The Metropolis acceptance ratio for the proposed $\alpha^*$ is\footnote{We
  log-transform the Metropolis acceptance ratio for better numerics.}

\begin{align}
  \log MH_\alpha &= \sum_i \Bigl[ (\alpha^* - \alpha)' W_{a_i} + \nonumber \\
                 & \log\left(\sum\limits_{j:j \in O_i} \exp(\alpha' W_j)\right) -
                   \log\left(\sum\limits_{j:j \in O_i} \exp(\alpha^{*\prime} W_j)\right) + \nonumber\\
                 & \log p(\alpha^*) - \log p(\alpha) \Bigr]
\end{align}

\subsection{Posterior of workers' preference $p(\beta|A, O, \alpha)$}

We propose a new $\bm{\beta}^*$ using a Normal, adaptive proposal distribution
similar to $\alpha$. Because $\bm{\beta}$ is high dimensional, with one set of
$\beta$ for each employer, in each MCMC iteration we randomly choose and update
only a part of $\bm{\beta}$.

The Metropolis acceptance ratio for the proposed $\beta$ is

\begin{align}
  \log MH_\beta &= \sum_i \left[ \sum_{j \in O_i} \beta_j^{*\prime}X_i - \beta_j^{\prime}X_i + \sum_{j} \log(1 + {\exp({\beta_j^{*\prime}X_i})) - \log(1 +  \exp(\beta_j^{\prime}X_i})) \right] \nonumber \\
                & + \log p(\bm{\beta}^*) - \log p(\bm{\beta})
\end{align}


\subsection{Posterior of $\beta$'s hyperparameters $\mu_{\beta}, \tau_{\beta}$}

As discussed above, the Bayesian approach to estimating our two-sided model
allows us to put a hierarchical structure on the preference parameter. Here, we
model firms' preference $\bm{\beta}$ as being drawn from the multivariate normal
distribution $MVN(\mu_{\beta}, \tau_{\beta})$, where $\mu_{\beta}$ is the mean
and $\tau_{\beta}$ is the precision.

When the prior $p(\beta)$ is also normal, we have a conjugate multivariate
normal model, where $\mu_{\beta}$ and $\tau_{\beta}$ are the parameters while
$\beta$ is considered the ``data''.

Since the model is conjugate, we can sample from the posterior of $\mu_{\beta}$
and $\tau_{\beta}$ with Gibbs sampling. Their full conditional distribution of
$\mu_{\beta}$ is:

\begin{align}
  p(\mu_{\beta}) &\sim MVN(\mu_0, \Sigma_0) \\
  p(\mu_{\beta} | \beta, \tau_{\beta}) &\sim MVN(m, V) \text{ where } \\
  V &= (\Sigma_0^{-1} + n \tau_{\beta})^{-1} \\
  m &= (\Sigma_0^{-1} + n \tau_{\beta})^{-1} (\Sigma_0^{-1}\mu_0 + n \tau_{\beta} \bar \beta)
\end{align}

The full conditional distribution of $\tau_{\beta}$ is:

\begin{align}
  p(\tau_{\beta}) &\sim \text{Wishart}(\nu_0, S_0^{-1}) \\
  p(\tau_{\beta} | \beta, \mu_{\beta}) &\sim \text{Wishart}(\nu, S^{-1}) \text{ where } \\
  \nu &= \nu_0 + n \\
  S^{-1} &= \left(S_0 + \sum (\beta - \mu_{\beta})(\beta - \mu_{\beta})^{\prime}\right)^{-1}
\end{align}

\section{Results for US labor data}

To be written ...
