\chapter{Two-Sided Matching Model}
\label{chap:model}

This section describes the behavioral model of many-to-one matching. To aid the
exposition, as an example for many-to-one matching I will use the labor market
where many workers can be matched with one firm.

We assume that the matching process in the labor market happens in two stage.
In the first stage, each firm evaluates each worker in the sample, then
decide whether to hire that worker or not. At the end of this stage, each worker
has a set of offers from firms, which we call his \textit{opportunity
  set}. In the second stage, each worker evaluates the firms in his
opportunity set and chooses the firm that he likes best. This constitutes a
final, observed match between a worker and a firm. This is a many-to-one
matching problem because a firm can make offers to multiple
workers, none, some, or all of which can be accepted by workers.

If we assume that firms and workers are utility-maximizing agents, at the end of
this process, no firm or worker would voluntarily change their final matches.
This property is called \textit{stability} in the game theoretic two-sided
matching literature. We want the model to have this property because real life
matching market tends to be stable as well. Indeed, \citep{Roth1992} show that
for any given set of preferences, a stable match always exist. Furthermore,
\citep{Roth2016} and \citep{Adachi2003} show that a decentralized market with
agents making independent, utility-maximizing decisions can also reach a stable
match by itself.

This stability property does not implies that the matches will not change.
Indeed, if actors' preference shift, their characteristics change, or new actors
enter the market, the matches will also change as a result of actors'
recalculating their utility and adjust their decisions. Therefore, since we are
estimating the actors' preference using only a snapshot of matching market, we
are making the assumption that on a systemic level, the average characteristics of the
actors and their preference remain sufficiently static for our estimates to be meaningful.

\subsection{The decision making of firms}

A firm $j$'s decision on whether to hire worker $i$ rests on two utility
functions. First, firm $j$'s utility for hiring worker $i$ is:

\begin{align}
U_j(i) &= \bm{\beta}_j' X_i + \epsilon_{1ij}
\end{align}

where

\begin{align*}
\beta_j &\text{ is a vector of firm $j$'s preference for worker characteristics} \\
x_i &\text{ is a vector of worker $i$'s measured values on those characteristics} \\
\epsilon_{1ij} &\text{ is the unobserved component that influences firm $j$'s utility}
\end{align*}

On the other hand, the utility of not hiring worker $i$ is:

\begin{align}
U_j(\neg i) &= b_j + \epsilon_{0ij}
\end{align}

where

\begin{align*}
b_j &\text{ is the baseline utility of firm $j$} \\
\epsilon_{0ij} &\text{ is the unobserved component that influences firm $j$'s utility}
\end{align*}

For each worker $i$, firm $j$ will make an offer to hire if $U_j(i) > U_j(\neg
i)$. Relevant worker characteristics (i.e. $X_i$) that a firm may consider are
age, education, or experience. The corresponding $\beta$'s represent the firm's
preference for these characteristics.

We make several assumptions that are standard in the discrete choice literature.
First, we assume a linear utility function. Second, we assume that the error
terms $\epsilon_{1ij}, \epsilon_{0ij}$ are uncorrelated. This assumption is most
likely not true: unobserved factors in one firm's utility are likely to share
some components with unobserved factors in another firm's utility, and thus
correlated. The hope is that we have modeled the observed portion of firm's
utility sufficiently well that the remaining unobserved factors are close to
white noise. In any case, this issue afflicts any application of discrete choice
models and is not unique to our case.\footnote{The discrete choice model has
  developed solutions for correlated error structure, such as nested logit,
  probit, and mixed logit, that can be applied here if we suspect that firms'
  unobserved utility is strong correlated.} 

Third, we assume that the as error terms $\epsilon_{1ij}, \epsilon_{0ij}$ follow
the Gumbel distribution, which has a slightly fatter tail the normal
distribution allowing for slightly more extreme variation in the unobserved
utility. In practice, the difference between using Gumbel and independent normal
error terms is small \citep{Train2009}. The choice of the Gumbel distribution is largely motivated by its
convenience since we can derive the probability of firm $j$ making an offer to worker $i$ as the familiar binomial logit form:

Also discuss IIA here

\begin{align}
Pr(o_{ij} = 1) &= Pr(U_j(i) > U_j(\neg i)) \\
&= Pr(\epsilon_{0ij} - \epsilon_{1ij} <  \bm{\beta}_j ' X_i - b_j) \\
&= \frac{\exp({\bm{\beta}_j'X_i})}{1 + \exp({\bm{\beta}_j'X_i})} \label{eq:prob_offer_ij}
\end{align}

\begin{align}
p(O_i | \bm{\beta}) &= \prod_{j \in O_i} p(o_{ij} = 1 | \bm{\beta}) \prod_{j \notin O_i} p(o_{ij} = 0 | \bm{\beta}) \\
&= \prod_{j \in O_i} \frac{\exp(\bm{\beta_j} ' X_i)}{1 + \exp(\bm{\beta_j}' X_i)}
 \prod_{j \notin O_i} \frac{1}{1 + \exp(\bm{\beta_j}' X_i)} \label{eq:conditional_probability_of_offer}
\end{align}

In our observed data, since we only observe the final matching of firms and countries, this opportunity set is unobserved. As will discuss, we use the Metropolis-Hastings algorithm to approximate the posterior distribution of the opportunity set.

\subsection{Firms' utility}

On the other side, for firm $i$, the utility of investing in country $j$ is:

\begin{align}
V_i(j) &= \alpha' W_{j} + v_{ij}
\end{align}

where

\begin{align*}
\alpha &\text{ is a vector of firms' preference for relevant characteristics of countries} \\
W_j &\text{ is a vector of country $j$ measured values on those characteristics} \\
v_{ij} &\text{ is the unobserved component that influences firm $i$'s utility}
\end{align*}

Firm $i$ evaluates all the countries that welcome it to invest and chooses the country that brings the highest utility. This choice of firms concludes the matching process, resulting in the observed final match between a firm and a country in our data.

In our model, relevant country characteristics can be: labor quality, level of development, and market size. Since all firms are considered having homogeneous preferences, $\alpha$ does not have a subscript $i$. The model can be easily extended so that there is heterogeneous preference among firms.

If $v_{ij}$ is modeled as having a Gumbel distribution, then the probability that firm $i$ will accept the offer of official $j$ out of all the offers in its opportunity set $O_i$ takes the multinomial logit form \citep{Cameron2005}:

\begin{align}
p(A_i = a_i | O_i, \alpha_i) = \frac{\exp(\alpha'W_{a_i})}{\sum\limits_{j:j \in O_i} \exp(\alpha'W_j)} \label{eq:conditional_probability_of_accept}
\end{align}

\section{Model Estimation}

Because the opportunity set is unobserved, we have to use MCMC to estimate it.

\begin{align}
U_j(i) &= \bm{\beta}_j' X_i + \epsilon_{1ij}
U_j(\neg i) &= b_j + \epsilon_{0ij}
V_i(j) &= \alpha' W_{j} + v_{ij}
\end{align}

\begin{align}
Pr(o_{ij} = 1) &= Pr(U_j(i) > U_j(\neg i)) \\
&= Pr(\epsilon_{0ij} - \epsilon_{1ij} <  \bm{\beta}_j ' X_i - b_j) \\
&= \frac{\exp({\bm{\beta}_j'X_i})}{1 + \exp({\bm{\beta}_j'X_i})} \label{eq:prob_offer_ij}
\end{align}

\begin{align}
p(O_i | \bm{\beta}) &= \prod_{j \in O_i} p(o_{ij} = 1 | \bm{\beta}) \prod_{j \notin O_i} p(o_{ij} = 0 | \bm{\beta}) \\
&= \prod_{j \in O_i} \frac{\exp(\bm{\beta_j} ' X_i)}{1 + \exp(\bm{\beta_j}' X_i)}
 \prod_{j \notin O_i} \frac{1}{1 + \exp(\bm{\beta_j}' X_i)} \label{eq:conditional_probability_of_offer}
\end{align}

\begin{align}
p(A_i = a_i | O_i, \alpha_i) = \frac{\exp(\alpha'W_{a_i})}{\sum\limits_{j:j \in O_i} \exp(\alpha'W_j)} \label{eq:conditional_probability_of_accept}
\end{align}

Joint likelihood:

\begin{align}
p(O, A, \alpha, \beta, \mu_{\beta}, \tau_{\beta}) &= p(A|O, \alpha) p(O|\beta) p(\alpha) p(\beta|\mu_{\beta}, \tau_{\beta}) p(\mu_{\beta}) p(\tau_{\beta}) 
\end{align}

\subsection{Updating the opportunity set}

Target distribution for a firm $i$ 

\begin{align}
p(O_i | A_i, \alpha, \bm{\beta}) &= \frac{p(O_i, A_i, \alpha, \bm{\beta})}{p(A_i, \alpha, \bm{\beta})}
\end{align}

\begin{align}
MH_O = \frac{p(O_i^* | A_i, \alpha, \bm{\beta})}{p(O_i | A_i, \alpha, \bm{\beta})} &= \frac{p(O_i^*, A_i, \alpha, \bm{\beta})}{p(A_i, \alpha, \bm{\beta})} \times \frac{p(A_i, \alpha, \bm{\beta})}{p(O_i, A_i, \alpha, \bm{\beta})} \\
&= \frac{p(O_i^*, A_i, \alpha, \bm{\beta})}{p(O_i, A_i, \alpha, \bm{\beta})} \\
&= \frac{p(A_i | O_i^*, \alpha)p(O_i^*|\bm{\beta})}{p(A_i | O_i, \alpha)p(O_i|\bm{\beta})} \label{eq:updateO_joint_dist_into_conditional_dist} \\
\end{align}

where the factorization of the likelihood in \eqref{eq:updateO_joint_dist_into_conditional_dist} is due to the fact that the acceptance of firm $i$ only depends on what is offered to it and what is its preference, $p(A_i | O_i^*, \alpha)$; what is offered to $i$ depends on the preferences of all countries, $p(O_i^* | \bm{\beta})$.

If we plug in \eqref{eq:conditional_probability_of_accept} and \eqref{eq:conditional_probability_of_offer}

\begin{align}
\frac{p(O_i^* | A_i, \alpha, \bm{\beta})}{p(O_i | A_i, \alpha, \bm{\beta})} &= \frac{\sum\limits_{j:j \in O_i} \exp(\alpha'W_j)}{\sum\limits_{j:j \in O_i} \exp(\alpha'W_j) + \exp(\alpha' W_{j^*})} \times \exp(\bm{\beta}_{j^*}'X_i)
\end{align}

where $j^*$ is the index of the newly sampled job. This is the case when the newly proposed job is not already offered, so it's added to the opportunity set.

When the newly proposed job is already offered, so it's removed from the opportunity set, we have

\begin{align}
\frac{p(O_i^* | A_i, \alpha, \bm{\beta})}{p(O_i | A_i, \alpha, \bm{\beta})} &= \frac{\sum\limits_{j:j \in O_i} \exp(\alpha'W_j)}{\sum\limits_{j:j \in O_i} \exp(\alpha'W_j) - \exp(\alpha' W_{j^*})} \times \exp(- \bm{\beta}_{j^*}'X_i)
\end{align}

\subsection{Updating firms' parameters, $\alpha$}

Target distribution:

\begin{align}
p(\alpha | A, O, \bm{\beta}) &= \frac{p(O, A, \alpha, \bm{\beta})}{p(A, O, \bm{\beta})}
\end{align}

We propose a new $\alpha^*$ using a symmetric proposal distribution that sample $\alpha^*$ in a box whose boundary is $\alpha^* \pm \epsilon_\alpha$

Metropolis-Hasting acceptance ratio:

\begin{align}
MH_\alpha = \frac{p(\alpha^* | A, O, \bm{\beta})}{p(\alpha | A, O, \bm{\beta})} &= \frac{p(A_i | O_i, \alpha^*)p(O_i|\bm{\beta})}{p(A_i | O_i, \alpha)p(O_i|\bm{\beta})} \\
&= \frac{p(A_i | O_i, \alpha^*)}{p(A_i | O_i, \alpha)} \label{eq:updatealpha_MHratio_final}
\end{align}

where \eqref{eq:updatealpha_MHratio_final} is due to the flat prior (so $\frac{p(\alpha^*)}{p(\alpha)}=1$) and the symmetric proposal distribution (so $\frac{p(\alpha^*|\alpha)}{p(\alpha|\alpha^*)} = 1$)

If we plug in \eqref{eq:conditional_probability_of_accept},

\begin{align}
MH_\alpha &= \prod_i \left[ \frac{\exp(\alpha^{*\prime} W_{a_i})}{\exp(\alpha' W_{a_i})} \times \frac{\sum\limits_{j:j \in O_i} \exp(\alpha' W_j)}{\sum\limits_{j:j \in O_i} \exp(\alpha^{*\prime}W_j)} \right] \\
&= \prod_i \left[ \exp(\epsilon_\alpha ' W_{a_i}) \times \frac{\sum\limits_{j:j \in O_i} \exp(\alpha' W_j)}{\sum\limits_{j:j \in O_i} \exp(\alpha^{*\prime}W_j)} \right]
\end{align}

Finally, we log transform the MH acceptance ratio for numerical stability.

\begin{align}
\log MH_\alpha &= \sum_i \left[ \epsilon_\alpha' W_{a_i} + \log\left(\sum\limits_{j:j \in O_i} \exp(\alpha' W_j)\right) - \log\left(\sum\limits_{j:j \in O_i} \exp(\alpha^{*\prime} W_j)\right) \right]
\end{align}

\subsection{Updating countries' parameters, \texorpdfstring{$\boldmath\beta$}{}}

Target distribution:

\begin{align}
p(\bm{\beta}|A, O, \alpha) &= \frac{p(O, A, \alpha, \bm{\beta})}{p(A, O, \alpha)}
\end{align}

We propose a new $\bm{\beta}^*$ using a symmetric proposal distribution that sample $\bm{\beta}^*$ in a box with side length $\epsilon_\beta$

Metropolis-Hasting acceptance ratio:

\begin{align}
MH_\beta = \frac{p(\beta^* | A, O, \alpha)}{p(\beta | A, O, \alpha)} &= \frac{p(A_i | O_i, \alpha)p(O_i|\bm{\beta}^*)p(\bm{\beta}^*|\mu_{\beta}, \tau_{\beta})}{p(A_i | O_i, \alpha)p(O_i|\bm{\beta})p(\bm{\beta}|\mu_{\beta}, \tau_{\beta})} \label{eq:updatebeta_MHratio_simplify} \\
&= \frac{p(O_i|\bm{\beta}^*)p(\bm{\beta}^*|\mu_{\beta}, \tau_{\beta})}{p(O_i|\bm{\beta})p(\bm{\beta}|\mu_{\beta}, \tau_{\beta})} \label{eq:updatebeta_MHratio_final}
\end{align}

where \eqref{eq:updatebeta_MHratio_simplify} is due to the flat prior on $\beta$ and the symmetric proposal distribution.

We plug in \eqref{eq:conditional_probability_of_offer},

\begin{align}
MH_\beta &= \prod_i \left[ \prod\limits_{j \in O_i}\frac{ \exp(\beta_j^{*\prime}X_i)}{ \exp(\beta_j^{\prime}X_i)} \times \prod\limits_{j}\frac{1 + \exp(\beta_j^{*\prime}X_i)}{1 + \exp(\beta_j^{\prime}X_i)} \right] \times \frac{MVN(\bm{\beta}^*|\mu_{\beta}, \tau_{\beta})}{MVN(\bm{\beta}|\mu_{\beta}, \tau_{\beta})} \\
  \log MH_\beta &= \sum_i \left[ \sum_{j \in O_i} \beta_j^{*\prime}X_i - \beta_j^{\prime}X_i + \sum_{j} \log(1 + {\exp({\beta_j^{*\prime}X_i})) - \log(1 +  \exp(\beta_j^{\prime}X_i})) \right] \\
 & + \log MVN(\bm{\beta}^*|\mu_{\beta}, \tau_{\beta}) - \log MVN(\bm{\beta}|\mu_{\beta}, \tau_{\beta}) \nonumber
\end{align}

In the MCMC implementation, since $\bm{\beta}$ is high dimensional, in each
step, we randomly update several $\beta$'s at one time.

\subsection{Update $\mu_{\beta}, \tau_{\beta}$}

Similar to a multivariate normal model, where $\beta$ is the ``data''.

\begin{align}
  p(\mu_{\beta}) &\sim MVN(\mu_0, \Sigma_0) \\
  p(\mu_{\beta} | \beta, \tau_{\beta}) &\sim MVN(m, V) \text{ where } \\
  V &= (\Sigma_0^{-1} + n \tau_{\beta})^{-1} \\
  m &= (\Sigma_0^{-1} + n \tau_{\beta})^{-1} (\Sigma_0^{-1}\mu_0 + n \tau_{\beta} \bar \beta)
\end{align}

