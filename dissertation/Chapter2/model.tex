\chapter{Two-Sided Matching Model}
\label{chap:model}

Here I present a behavioral model of the two-sided matching market, focusing on
the case of many-to-one matching. For easier exposition, throughout the chapter
I will use the example of the labor market, where many workers can be matched to
one firm, to demonstrate the features of this model.

In this model, there are two separate sets of actors: firms and workers. There
are two sub-components of the model, happening sequentially. In the first stage, each firm
evaluates each worker in the sample, then decide whether to extend an offer or
not. In the second stage, 

meaning that the model rests on the behavioral assumption that
people make choices by maximizing their utilities. The parameters estimated will
be the preference of people, and can be interpreted as how much they value
different features. Since utility is unit-less, it won't have a scale, we only
care about the relative values of different parameters.
Cite McFadden here

- one sided model won't be able to distinguish the effect between preference and opportunity
- follows the tradition of discrete choice, in which actors choose from a set of
finite and discrete alternatives (firms choosing from workers, workers choosing
offers). Here we do make the assumption that the sample is representative of the
population, i.e. the workers in the sample are also the workers that are
presented to firm

This section describes the two-sided matching model, a behavioral model in the
sense that actors which is the
behavioincluding the utilities of countries' officials and of MNCs. Then, the matching process is a natural consequence of actors' choosing the best option available to them.

This model is a parametric version of the matching game by Roth 

Describe the model: there are two sets of actors, one is employers, the other is
employee. Employer consider every single worker, determine the utility of hiring
him over not hiring him. This makes the assumption that the hiring decision does
not depend on whether other candidates will accept the offer (substitutability).
It also assumes that firm can theoretically hire all workers in the pool if they
accept (assuming that there is no quota).

IIA the ratio of the probabilities of choosing between two alternatives is
independent of the attributes of all other alternatives. (This applies to the
worker side).

Stress that this is only one way to reach stable matching

For the case of FDI, if we assume that all firms have the same preference, then
even if we don't include the firms who decide not to invest, that's fine.

\subsection{Officials' Utility}

Following \citet{Logan1998}, we consider the utility function of two set of
actors, firms and workers. For firm $j$, the utility of hiring worker $i$ is:

\begin{align}
U_j(i) &= \bm{\beta}_j' X_i + \epsilon_{1ij}
\end{align}

where

\begin{align*}
\beta_j &\text{ is a vector of official $j$'s preference for relevant characteristics of firms} \\
x_i &\text{ is a vector of firm $i$'s measured values on those characteristics} \\
\epsilon_{1ij} &\text{ is the unobserved component that influences official $j$'s utility}
\end{align*}

On the other hand, the utility of not having firm $i$ investing is:

\begin{align}
U_j(\neg i) &= b_j + \epsilon_{0ij}
\end{align}

where

\begin{align*}
b_j &\text{ is the baseline utility of official $j$ without any firm investing} \\
\epsilon_{0ij} &\text{ is the component that influences official $j$'s utility}
\end{align*}

For each firm $i$, official $j$ will make an offer to invest if $U_j(i) > U_j(\neg i)$. Relevant firm characteristics (i.e. $X_i$) that the official may consider are: technological intensity, number of jobs, and size of capital. The corresponding $\beta$'s represent the official's preference for these characteristics.

Following the discrete choice literature, we model $\epsilon_{1ij}, \epsilon_{0ij}$ as having the Gumbel distribution. Then, the probability of official $j$ making an offer to firm $i$ takes the familiar binomial logit form:

\begin{align}
Pr(o_{ij} = 1) &= Pr(U_j(i) > U_j(\neg i)) \\
&= Pr(\epsilon_{0ij} - \epsilon_{1ij} <  \bm{\beta}_j ' X_i - b_j) \\
&= \frac{\exp({\bm{\beta}_j'X_i})}{1 + \exp({\bm{\beta}_j'X_i})} \label{eq:prob_offer_ij}
\end{align}

\begin{align}
p(O_i | \bm{\beta}) &= \prod_{j \in O_i} p(o_{ij} = 1 | \bm{\beta}) \prod_{j \notin O_i} p(o_{ij} = 0 | \bm{\beta}) \\
&= \prod_{j \in O_i} \frac{\exp(\bm{\beta_j} ' X_i)}{1 + \exp(\bm{\beta_j}' X_i)}
 \prod_{j \notin O_i} \frac{1}{1 + \exp(\bm{\beta_j}' X_i)} \label{eq:conditional_probability_of_offer}
\end{align}

In our observed data, since we only observe the final matching of firms and countries, this opportunity set is unobserved. As will discuss, we use the Metropolis-Hastings algorithm to approximate the posterior distribution of the opportunity set.

\subsection{Firms' utility}

On the other side, for firm $i$, the utility of investing in country $j$ is:

\begin{align}
V_i(j) &= \alpha' W_{j} + v_{ij}
\end{align}

where

\begin{align*}
\alpha &\text{ is a vector of firms' preference for relevant characteristics of countries} \\
W_j &\text{ is a vector of country $j$ measured values on those characteristics} \\
v_{ij} &\text{ is the unobserved component that influences firm $i$'s utility}
\end{align*}

Firm $i$ evaluates all the countries that welcome it to invest and chooses the country that brings the highest utility. This choice of firms concludes the matching process, resulting in the observed final match between a firm and a country in our data.

In our model, relevant country characteristics can be: labor quality, level of development, and market size. Since all firms are considered having homogeneous preferences, $\alpha$ does not have a subscript $i$. The model can be easily extended so that there is heterogeneous preference among firms.

If $v_{ij}$ is modeled as having a Gumbel distribution, then the probability that firm $i$ will accept the offer of official $j$ out of all the offers in its opportunity set $O_i$ takes the multinomial logit form \citep{Cameron2005}:

\begin{align}
p(A_i = a_i | O_i, \alpha_i) = \frac{\exp(\alpha'W_{a_i})}{\sum\limits_{j:j \in O_i} \exp(\alpha'W_j)} \label{eq:conditional_probability_of_accept}
\end{align}

\section{Model Estimation}

Because the opportunity set is unobserved, we have to use MCMC to estimate it.

\begin{align}
U_j(i) &= \bm{\beta}_j' X_i + \epsilon_{1ij}
U_j(\neg i) &= b_j + \epsilon_{0ij}
V_i(j) &= \alpha' W_{j} + v_{ij}
\end{align}

\begin{align}
Pr(o_{ij} = 1) &= Pr(U_j(i) > U_j(\neg i)) \\
&= Pr(\epsilon_{0ij} - \epsilon_{1ij} <  \bm{\beta}_j ' X_i - b_j) \\
&= \frac{\exp({\bm{\beta}_j'X_i})}{1 + \exp({\bm{\beta}_j'X_i})} \label{eq:prob_offer_ij}
\end{align}

\begin{align}
p(O_i | \bm{\beta}) &= \prod_{j \in O_i} p(o_{ij} = 1 | \bm{\beta}) \prod_{j \notin O_i} p(o_{ij} = 0 | \bm{\beta}) \\
&= \prod_{j \in O_i} \frac{\exp(\bm{\beta_j} ' X_i)}{1 + \exp(\bm{\beta_j}' X_i)}
 \prod_{j \notin O_i} \frac{1}{1 + \exp(\bm{\beta_j}' X_i)} \label{eq:conditional_probability_of_offer}
\end{align}

\begin{align}
p(A_i = a_i | O_i, \alpha_i) = \frac{\exp(\alpha'W_{a_i})}{\sum\limits_{j:j \in O_i} \exp(\alpha'W_j)} \label{eq:conditional_probability_of_accept}
\end{align}

Joint likelihood:

\begin{align}
p(O, A, \alpha, \beta, \mu_{\beta}, \tau_{\beta}) &= p(A|O, \alpha) p(O|\beta) p(\alpha) p(\beta|\mu_{\beta}, \tau_{\beta}) p(\mu_{\beta}) p(\tau_{\beta}) 
\end{align}

\subsection{Updating the opportunity set}

Target distribution for a firm $i$ 

\begin{align}
p(O_i | A_i, \alpha, \bm{\beta}) &= \frac{p(O_i, A_i, \alpha, \bm{\beta})}{p(A_i, \alpha, \bm{\beta})}
\end{align}

\begin{align}
MH_O = \frac{p(O_i^* | A_i, \alpha, \bm{\beta})}{p(O_i | A_i, \alpha, \bm{\beta})} &= \frac{p(O_i^*, A_i, \alpha, \bm{\beta})}{p(A_i, \alpha, \bm{\beta})} \times \frac{p(A_i, \alpha, \bm{\beta})}{p(O_i, A_i, \alpha, \bm{\beta})} \\
&= \frac{p(O_i^*, A_i, \alpha, \bm{\beta})}{p(O_i, A_i, \alpha, \bm{\beta})} \\
&= \frac{p(A_i | O_i^*, \alpha)p(O_i^*|\bm{\beta})}{p(A_i | O_i, \alpha)p(O_i|\bm{\beta})} \label{eq:updateO_joint_dist_into_conditional_dist} \\
\end{align}

where the factorization of the likelihood in \eqref{eq:updateO_joint_dist_into_conditional_dist} is due to the fact that the acceptance of firm $i$ only depends on what is offered to it and what is its preference, $p(A_i | O_i^*, \alpha)$; what is offered to $i$ depends on the preferences of all countries, $p(O_i^* | \bm{\beta})$.

If we plug in \eqref{eq:conditional_probability_of_accept} and \eqref{eq:conditional_probability_of_offer}

\begin{align}
\frac{p(O_i^* | A_i, \alpha, \bm{\beta})}{p(O_i | A_i, \alpha, \bm{\beta})} &= \frac{\sum\limits_{j:j \in O_i} \exp(\alpha'W_j)}{\sum\limits_{j:j \in O_i} \exp(\alpha'W_j) + \exp(\alpha' W_{j^*})} \times \exp(\bm{\beta}_{j^*}'X_i)
\end{align}

where $j^*$ is the index of the newly sampled job. This is the case when the newly proposed job is not already offered, so it's added to the opportunity set.

When the newly proposed job is already offered, so it's removed from the opportunity set, we have

\begin{align}
\frac{p(O_i^* | A_i, \alpha, \bm{\beta})}{p(O_i | A_i, \alpha, \bm{\beta})} &= \frac{\sum\limits_{j:j \in O_i} \exp(\alpha'W_j)}{\sum\limits_{j:j \in O_i} \exp(\alpha'W_j) - \exp(\alpha' W_{j^*})} \times \exp(- \bm{\beta}_{j^*}'X_i)
\end{align}

\subsection{Updating firms' parameters, $\alpha$}

Target distribution:

\begin{align}
p(\alpha | A, O, \bm{\beta}) &= \frac{p(O, A, \alpha, \bm{\beta})}{p(A, O, \bm{\beta})}
\end{align}

We propose a new $\alpha^*$ using a symmetric proposal distribution that sample $\alpha^*$ in a box whose boundary is $\alpha^* \pm \epsilon_\alpha$

Metropolis-Hasting acceptance ratio:

\begin{align}
MH_\alpha = \frac{p(\alpha^* | A, O, \bm{\beta})}{p(\alpha | A, O, \bm{\beta})} &= \frac{p(A_i | O_i, \alpha^*)p(O_i|\bm{\beta})}{p(A_i | O_i, \alpha)p(O_i|\bm{\beta})} \\
&= \frac{p(A_i | O_i, \alpha^*)}{p(A_i | O_i, \alpha)} \label{eq:updatealpha_MHratio_final}
\end{align}

where \eqref{eq:updatealpha_MHratio_final} is due to the flat prior (so $\frac{p(\alpha^*)}{p(\alpha)}=1$) and the symmetric proposal distribution (so $\frac{p(\alpha^*|\alpha)}{p(\alpha|\alpha^*)} = 1$)

If we plug in \eqref{eq:conditional_probability_of_accept},

\begin{align}
MH_\alpha &= \prod_i \left[ \frac{\exp(\alpha^{*\prime} W_{a_i})}{\exp(\alpha' W_{a_i})} \times \frac{\sum\limits_{j:j \in O_i} \exp(\alpha' W_j)}{\sum\limits_{j:j \in O_i} \exp(\alpha^{*\prime}W_j)} \right] \\
&= \prod_i \left[ \exp(\epsilon_\alpha ' W_{a_i}) \times \frac{\sum\limits_{j:j \in O_i} \exp(\alpha' W_j)}{\sum\limits_{j:j \in O_i} \exp(\alpha^{*\prime}W_j)} \right]
\end{align}

Finally, we log transform the MH acceptance ratio for numerical stability.

\begin{align}
\log MH_\alpha &= \sum_i \left[ \epsilon_\alpha' W_{a_i} + \log\left(\sum\limits_{j:j \in O_i} \exp(\alpha' W_j)\right) - \log\left(\sum\limits_{j:j \in O_i} \exp(\alpha^{*\prime} W_j)\right) \right]
\end{align}

\subsection{Updating countries' parameters, \texorpdfstring{$\boldmath\beta$}{}}

Target distribution:

\begin{align}
p(\bm{\beta}|A, O, \alpha) &= \frac{p(O, A, \alpha, \bm{\beta})}{p(A, O, \alpha)}
\end{align}

We propose a new $\bm{\beta}^*$ using a symmetric proposal distribution that sample $\bm{\beta}^*$ in a box with side length $\epsilon_\beta$

Metropolis-Hasting acceptance ratio:

\begin{align}
MH_\beta = \frac{p(\beta^* | A, O, \alpha)}{p(\beta | A, O, \alpha)} &= \frac{p(A_i | O_i, \alpha)p(O_i|\bm{\beta}^*)p(\bm{\beta}^*|\mu_{\beta}, \tau_{\beta})}{p(A_i | O_i, \alpha)p(O_i|\bm{\beta})p(\bm{\beta}|\mu_{\beta}, \tau_{\beta})} \label{eq:updatebeta_MHratio_simplify} \\
&= \frac{p(O_i|\bm{\beta}^*)p(\bm{\beta}^*|\mu_{\beta}, \tau_{\beta})}{p(O_i|\bm{\beta})p(\bm{\beta}|\mu_{\beta}, \tau_{\beta})} \label{eq:updatebeta_MHratio_final}
\end{align}

where \eqref{eq:updatebeta_MHratio_simplify} is due to the flat prior on $\beta$ and the symmetric proposal distribution.

We plug in \eqref{eq:conditional_probability_of_offer},

\begin{align}
MH_\beta &= \prod_i \left[ \prod\limits_{j \in O_i}\frac{ \exp(\beta_j^{*\prime}X_i)}{ \exp(\beta_j^{\prime}X_i)} \times \prod\limits_{j}\frac{1 + \exp(\beta_j^{*\prime}X_i)}{1 + \exp(\beta_j^{\prime}X_i)} \right] \times \frac{MVN(\bm{\beta}^*|\mu_{\beta}, \tau_{\beta})}{MVN(\bm{\beta}|\mu_{\beta}, \tau_{\beta})} \\
  \log MH_\beta &= \sum_i \left[ \sum_{j \in O_i} \beta_j^{*\prime}X_i - \beta_j^{\prime}X_i + \sum_{j} \log(1 + {\exp({\beta_j^{*\prime}X_i})) - \log(1 +  \exp(\beta_j^{\prime}X_i})) \right] \\
 & + \log MVN(\bm{\beta}^*|\mu_{\beta}, \tau_{\beta}) - \log MVN(\bm{\beta}|\mu_{\beta}, \tau_{\beta}) \nonumber
\end{align}

In the MCMC implementation, since $\bm{\beta}$ is high dimensional, in each
step, we randomly update several $\beta$'s at one time.

\subsection{Update $\mu_{\beta}, \tau_{\beta}$}

Similar to a multivariate normal model, where $\beta$ is the ``data''.

\begin{align}
  p(\mu_{\beta}) &\sim MVN(\mu_0, \Sigma_0) \\
  p(\mu_{\beta} | \beta, \tau_{\beta}) &\sim MVN(m, V) \text{ where } \\
  V &= (\Sigma_0^{-1} + n \tau_{\beta})^{-1} \\
  m &= (\Sigma_0^{-1} + n \tau_{\beta})^{-1} (\Sigma_0^{-1}\mu_0 + n \tau_{\beta} \bar \beta)
\end{align}

