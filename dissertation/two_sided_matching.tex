\chapter{Two-sided matching}

Much of our social, economic, and political life is governed by two-sided
matching markets. In these matching markets, actors from two disjoint sets
evaluate the characteristics of someone on the other side and voluntarily form a
match if both deem each other satisfactory.\footnote{Throughout the
  dissertation, I use ``two-sided matching market'' and ``matching market''
  interchangeably. On the other hand, note that a two-sided market is not
  necessarily a matching market \citep{Rysman2009}.} Marriage is a prominent
example of such matching process. Others include the matching between firms and
workers, federal judges and law clerks, the \textit{formateur} of a coalition
government and other minority parties, or countries and multinational
corporations (MNCs) that are looking for a location to invest.

Two-sided matching market is substantively consequential because it often
involves scarce, indivisible goods, such as life commitment to a marital partner
or political allegiance in a coalition government. It is also intellectually
interesting because the market outcome depends on the actions of both sides,
demanding a different analytical approach from what's used for one-sided
markets.

This chapter will proceed as follows. First, I discuss the game theory
literature of two-sided matching models, where much of the terminology and
insight originate. I will highlight key results that are relevant to our goal of
estimating actors' preference in matching markets. Second, I examine existing
empirical studies of matching markets. Where existing studies have not taken
into account the market's two-sided nature, I discuss how doing so can improve
our understanding of the subject area. Where existing studies do model the
two-sided dynamics, I discuss how they may or may not be used to study subjects
that political scientists are interested in.

\section{Game theory models of matching markets}
\label{sec:game_theory}

\citet{Gale1962} was the first to study the matching market, using marriage as
an example. In this market, there are two finite and disjoint sets of actors:
men and women. Each man has preferences over the women, and vice versa. Each
man's preference can be represented as an ordered list, ranking each woman based
on how much he likes her.

The outcome of this market is a set of marriages, with none of some of people
prefer to remain single. We call such a set of marriages a \textit{matching}
$\mu$, which is a one-to-one function that matches a man with a woman. We refer
to $\mu(x)$ as the \textit{mate} of $x$. For convenience, we say that if an
individual decides to remain single, they are matched with themselves.

We define a matching $\mu$ as \textit{stable} if it cannot be improved by any
individual or any pair of agents. A matching can be improved in two ways. First,
an individual may prefer to remain single than to be matched with his or her
mate $\mu(x)$ under the current matching $\mu$. Second, a man and a woman may
prefer to be with one another rather than whom they are currently matched with.
Therefore, if a matching is stable, no one has a better option than their
current situation.

The first key result from the game theory literature is that for any set of
preference, there always exists a stable matching \citep{Gale1962}. The proof is
constructive, describing the ``deferred acceptance'' procedure that is
guaranteed to produce a stable matching.\footnote{The ``deferred acceptance''
  procedure works as follows. In the first stage, every man proposes to his
  preferred mate. Every woman rejects all of her suitors except the one that she
  most prefers. However, she does not yet accept her (so far) favorite suitor,
  but keeps him along. In the second stage, every man that was rejected in the
  previous round proposes to his second choice. Every woman then picks her
  favorite from the set of new proposers and the man she keeps along from the
  previous round. The procedure continues until there is no longer any woman
  that is unmatched, at which point women finally accept their current favorite
  choices. (This procedure is called \textit{deferred acceptance} to capture the
  fact that women defer accepting her favorite choice until the last round in
  case better options become available.) The resulting match is stable because,
  throughout the procedure, every woman has received all the offers that would
  have been made to her, and she has chosen her favorite among all of those
  offers. If there were any other man that she would prefer to her current
  match, that man would not have been available to her. Therefore, the final
  match cannot be further improved by any man or woman.} This result provides
some justifications for us to assume that the matching we observe in real
matching market is stable, and that the agents' utility cannot be further
improved. Our empirical model of two-sided matching markets thus needs to
describe a process that produces a stable matching.

While a central coordinator employing the \textit{deferred acceptance} algorithm
is guaranteed to come up with a stable matching, it is unclear whether
decentralized markets, such as the labor market or the FDI market, would be able
to reach this outcome by themselves.\footnote{The deferred acceptance procedure
  was used in the market for US medical residency with enthusiastic
  participation from medical students and hospitals. The high participation rate
  indicates that the matching produced is stable enough to entice students and
  hospitals away from arranging their own matches outside of the centralized
  market.} The second key result from the game theory literature is that stable
matching in decentralized matching market is indeed possible, even likely. For
example, \citet{Roth2016} show that, starting from an arbitrary matching, the
market can converge to a stable matching with probability 1 if we allow random
blocking pairs, i.e. two individuals that are not matched but prefer each other
to their current match, to break off and form their own match. In addition,
\citet{Adachi2003} shows that a random search process, in which pairs of man and
woman randomly meet and decide whether each other is better than their current
mates, will converge towards a stable matching if the search cost is
negligible.\footnote{In this model, searching has a time cost. Thus, negligible
  search cost is modeled as agents having a time discount close to 1.} These
results further suggest that the matching we observe in decentralized markets is
likely stable. Therefore, our empirical model of matching markets to describe a
process that produces a stable matching.

The third key result is that all conclusions regarding the one-to-one matching
market (e.g. marriage) generalize to the many-to-one matching market (e.g.
college admission, labor market), albeit requiring additional assumptions
\citet{Roth1992}. One important assumption is that firms treat workers as
substitutes, not complements. In other words, firms never regret hiring a worker
even if another worker is no longer available. Therefore, when we conduct
empirical analysis of many-to-one markets, we should focus on markets where
agents have such ``substitutable preference.'' Otherwise, a stable matching is
not guaranteed, agents' utility functions are interdependent, and it becomes
unclear what kind of matching process our empirical model should approximate.

\section{Empirical models of matching markets}

The game theory literature takes the agents' preference as given and proves the
existence of a stable matching. In contrast, empirical models of matching
markets takes the observed matching as given and attempt to estimate the agents'
preference.

Unfortunately, most extant empirical models fail to adequately account for the
structure of a two-sided matching market. Often, researchers simply analyze the
market from one side, e.g. estimating a firm's preference by looking at the type
of workers it hires. This approach does not take into account the fact that a
match depends not only on the agent's preference but also his opportunity. For
example, a farm may prefer to hire highly-educated workers but cannot do so
because highly-educated workers do not want to work on farms. Modeling this
interaction between preference and opportunity is the key contribution of this
dissertation.

Alternatively, some researchers measure agents' preferences by surveying them
directly \citep{Posner2001, Sprecher1994}. While this approach circumvents the
need to disentangle preference and opportunity, it can only measure agents'
\textit{stated} preference. In addition, such surveys require a high data
collection effort while data on final matching (e.g. married couples, workers'
current job, country location of MNCs) are widely available. This dissertation
aims to make use of such available data to estimate agents' \textit{revealed}
preference.

Below I discuss existing empirical models of matching markets. First, I discuss
two markets of interest to political scientists: the US federal clerkship market
and the ``market'' for forming a coalition government. Researchers in both
subject areas have not approached the problem with an empirical model that
adequately captures its two-sided dynamics.

Second, I examine models from other disciplines that do take into account the
two-sided dynamics of matching markets. I start with machine learning models
applied to online marketplaces and dating sites. Then, I discuss the statistical
models of the labor market \citep{Logan1996} and the marriage market
\citep{Logan2008}, which are most relevant to our goal of estimating agents'
preference based on observed match data. These statistical models serve as the
foundation of my empirical approach.

\subsection{US federal clerkship market}

In the US, graduates at top law schools vie for the best federal clerkships
every year. These temporary, one-to-two-year positions are the launching pad for
Supreme Court clerkships, prestigious teaching jobs, or employment at top law
firms. On the other side, federal judges also compete for the best law
graduates, who help reduce the judges' workload from copy-editing to drafting
opinions \citep{Gulati2016, Posner2001}. Because the first clerkship tends to
have an outsized ideological influence on law graduates, this matching market
has important implications for the polarization of the judicial branch
\citep{Ditslear2001, Liptak2007}.

The market for US federal clerkship has been noted as a classic case of a
two-sided market. Clerks look for positions that provide not only prestige and
connection but also comfortable quality of life \citep{Posner2001}. Judges
select law graduates based on not only academic credentials but also, some
argue, ideology, gender, and race \citep{Slotnick1984}. This market also suffers
from strategic behavior emblematic of a matching market, such as offers being
made aggressively early and with a short time to accept \citep{Posner2001,
  Posner2007}.

One approach to estimating the preference of agents in this market is to survey
clerks and judges directly \citep{Peppers2008}. However, as discussed, this
approach only measures stated preference, which is likely to suffer from social
desirability bias when it comes to dimensions that we care about most such as
matching based on ideology, gender, or race.

Other approaches estimate revealed preference by using observed hiring outcome.
However, no existing study has properly taken into account the two-sided nature
of the market, thus confusing the effects of preference and opportunity. For
example, \citet{Bonica2017} use political contribution data (DIME dataset) to
measure political ideology, then correlate the ideology of the hiring judge and
the ideology of his clerks. This approach does not take into account the pool of
applicants, leading to conclusions such as conservative judges hire more liberal
clerks than conservative clerks \citep[31]{Bonica2017}. This curious finding has
a potentially simple explanation: the pool of top law graduates tend to be
overwhelmingly liberal, leaving conservative judges with no choice. Despite
this issue, the authors proceed to measure judges' ideology by taking the
average of their clerks' ideology. Without taking the pool of applicants into
account, they may wrongly conclude that conservative judges are more liberal
than they actually are.

In another approach, \citet{Rozema2016} model the process as a discrete choice
problem, in which clerks are differentiated products that Supreme Court justices
select to maximize their utilities. Their model does not consider what clerks
think about the offer because of their focus on Supreme Court clerkships, whose
unparalleled prestige ensures that any offer made will be accepted. However, if
we want to extend the model to the broader market of federal clerkship, such
assumption is untenable.

\subsection{The market for forming a coalition government}

Besides election, government formation is the most consequential political
process in determining which government people are subject to. Most extant
studies of government formation are either game theoretic models or thick,
``inside-the-Beltway'' narratives. Potential advances can be made if we consider
government formation as a many-to-one matching market, with the
\textit{formateur} party on one side and other minority parties on the
other.\footnote{The \textit{formateur} party could be the one with the
  procedural power to set up the coalition, e.g. the incumbent party, or the
  largest party in established coalitions.}

A two-sided matching model of government formation would complement the game
theory literature that models politicians as policy-seeking (as opposed to
office-seeking) \citep{Laver1998}. When politicians are policy-seeking, parties
have policy positions that can be modeled as their characteristics. Then,
parties choose one another to form a coalition based on their policy positions,
akin to men and women choosing one another to form a marriage based on their
height or income.\footnote{In contrast, when politicians are office-seeking, the
  only coin of the realm is the number of legislative seats that a party
  controls. It determines both the inclusion of the party in the government and
  its portfolio allocation. In this framework, concepts like power indices and
  dominant parties are all about how parties can bring its controlled seats to a
  coalition to turn it into a winning coalition.} As the game theory literature
suggests, ideologically compact coalitions are more valuable because they entail
a smaller cost in terms of policy compromises \citep{DeSwaan1973}. With the empirical
matching model, we can test if parties do indeed prefer others that are
ideologically close to themselves.

In addition, an advantage of the two-sided matching approach is its ability to
consider multidimensional policy spaces. By considering a party's positions on
various policies as their covariates, we would be able to estimate parties'
relative preference for ideological proximity across policy dimensions.

\subsection{The FDI market}

To be introduced here, or kept until its own empirical chapter?

\subsection{Recommender system for online two-sided markets}

In recent years, the Internet underwent a proliferation of two-sided matching
markets such as online marketplaces (e.g. AirBnB), dating sites (e.g. eHarmony),
or job board (e.g. Elance). To help their users discover a match quicker, these
sites often build a recommender system that suggests potential
matches.\footnote{To clarify, the term ``recommender system'' typically refers
  to systems that recommend items to users based on the reviews of users like
  them. That is not what we are discussing here. Instead, we focus on matching
  markets where the recommender system recommends users to one another.} To
maximize user engagement and profitability, these sites are incentivized to make
recommendations that resemble a stable matching so that their users get the best
match possible. And fo find the stable matching, they have to first estimate the
preferences of their users.

While most of these algorithms are proprietary, some academic publications have
addressed this problem. An interesting approach is the paper by \citet{Tu2014},
which uses the Latent Dirichlet allocation (LDA) model to uncover the latent
types of users based on their activities on an online dating
platform.\footnote{Besides \citet{Tu2014}, \citet{Hitsch2010, Goswami2014} are
  two other attempts to estimate users' preference in online matching markets.
  However, these papers take a simple one-sided approach, ignoring the interplay
  between preference and opportunity. Therefore, I don't discuss them further
  here.} In the original application of LDA model in topic modeling, each
document is a mixture of latent topics, and each topic is a distribution over
words. In this application, each user is a mixture of latent ``types,'' and each
type is a distribution signifying relative preference over various mates'
features. For example, the ``outdoor type'' may have higher preference for
athleticism or dog ownership over other traits.

While the LDA model works well for the online dating market, it is not
applicable to most social science problems for two reasons. First, this model
requires data of users reaching out to multiple partners rather than just the
final match. Second, while the LDA model uncovers users' latent types, most
social scientists want to estimate the preference of specific, known types (e.g.
how different regime types may prefer different characteristics of an MNC).

\subsection{Two-sided models for the labor and marriage markets}

To be introduced here, or in their own chapters?

\section{Conclusion}

Roadmap for the rest of the dissertation